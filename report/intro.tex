\section{Présentation de l'entreprise}


    \subsection{Description générale}
        \subsubsection{Activités de l'entreprise et historique}
            \paragraph{}
            Quelques chiffres

            \paragraph{}

        \subsubsection{localisation}
            \paragraph{}
            Ses domaines d'activités.
            \paragraph{}

        \subsubsection{Organisation}
            \paragraph{}
            Ses domaines d'activités.
            \paragraph{}

            Son pôle de recherche.
            \paragraph{}
            Ses compétences, son marché.
        \subsubsection{Présentation générale du service}
            \paragraph{}
La Direction des Services de la Navigation Aérienne est chargée de rendre le service de navigation aérienne pour l’Etat français. A ce titre, la DSNA est responsable de rendre les services de circulation aérienne, d’information aéronautique et d’alerte sur le territoire national et ceux d’outre-mer (DOM, TOM , POM). La DSNA s’appuie sur deux directions pour exécuter cette mission:
\begin{itemize}
\item La Direction des opérations ou DO,
\item la Direction de la Technique et de l’Innovation ou DTI.
\end{itemize}
La DO est l’acteur opérationnel du contrôle aérien tandis que la DTI est chargé du volet technique. Celui-ci consiste à réaliser ou acquérir les systèmes qui participent à l’exercice du contrôle aérien. Il s’agit de systèmes informatiques permettant d’assister le contrôleur dans ses activités, de chaines radios pour communiquer avec les aéronefs, de systèmes de traitement de l’information météorologique…
            \paragraph{}
La DTI réalise également de nombreuses études pour traiter les besoins des utilisateurs et les évolutions réglementaires. La DTI réalise le déploiement et le support opérationnel des systèmes qu’elle acquiert ou réalise. 
            \paragraph{}
Enfin la DTI fait viser ses systèmes, procédures et formation par l’autorité de surveillance nationale (Direction de la Sécurité de l'Aviation Civile ou DSAC).
            \paragraph{}
La DTI est structurée en domaines qui sont chacun en charge de plusieurs pôles de compétences :
\begin{itemize}
\item Recherche \& développement, R et D
\item Exigences opérationnelles des systèmes, EOS
\item Gestion du trafic aérien, ATM
\item Communication, navigation, surveillance, CNS
\item Déploiement et Support Opérationnel, DSO
\end{itemize}
            \paragraph{}
Chaque pôles qui couvre un ensemble de fonctions et d’expertises.
Pole ATM/VIG :
Le pôle « Vol et information générale » (VIG) est responsable de la maîtrise d’ouvrage systèmes de traitement des plans de vol et informations générales, à ce titre, le pôle assure le suivi industriel de leur réalisation ou de leur acquisition. Le pôle VIG est également chargé de leur maintien en condition opérationnelles lorsqu’ils sont déployés.
Le pôle ATM/VIG est notamment responsable de la maîtrise d’ouvrage de systèmes déployés en outre-mer. L’aéroport de Tahiti (Polynésie française) a récemment été modernisé avec un système entièrement acquis auprès d’un industriel, couplé à un radar dans le cadre du projet TIARE, qui s’est terminé en 2009.

    \subsubsection{Les partenaires}
        \paragraph{}
        Tissu local, sous-traitant.
        \paragraph{}
        Relations au plan local, nationnal...

    \subsection{Situation géographique}
        \paragraph{}
        Son emplacement.

    % ...

\newpage

\section{Environnement de travail}
% ...
