\section{sujet du stage}
    Copier collé du sujet
    
\section{Présentation de l’environnement}


    %\subsection{Description générale}
        %\subsubsection{Activités de l'entreprise et historique}
            %\paragraph{}
            %Quelques chiffres

            %\paragraph{}

        %\subsubsection{localisation}
            %\paragraph{}
            %Ses domaines d'activités.
            %\paragraph{}

        %\subsubsection{Organisation}
            %\paragraph{}
            %Ses domaines d'activités.
            %\paragraph{}

            %Son pôle de recherche.
            %\paragraph{}
            %Ses compétences, son marché.
    \subsection{DSNA/DTI}
La Direction des Services de la Navigation Aérienne est chargée de rendre le service de navigation aérienne pour l’Etat français. A ce titre, la DSNA est responsable de rendre les services de circulation aérienne, d’information aéronautique et d’alerte sur le territoire national et ceux d’outre-mer (DOM, TOM , POM). La DSNA s’appuie sur deux directions pour exécuter cette mission:
\begin{itemize}
\item La Direction des opérations ou DO,
\item la Direction de la Technique et de l’Innovation ou DTI.
\end{itemize}
La DO est l’acteur opérationnel du contrôle aérien tandis que la DTI est chargé du volet technique. Celui-ci consiste à réaliser ou acquérir les systèmes qui participent à l’exercice du contrôle aérien. Il s’agit de systèmes informatiques permettant d’assister le contrôleur dans ses activités, de chaines radios pour communiquer avec les aéronefs, de systèmes de traitement de l’information météorologique…

La DTI réalise également de nombreuses études pour traiter les besoins des utilisateurs et les évolutions réglementaires. La DTI réalise le déploiement et le support opérationnel des systèmes qu’elle acquiert ou réalise. 

Enfin la DTI fait viser ses systèmes, procédures et formation par l’autorité de surveillance nationale (Direction de la Sécurité de l'Aviation Civile ou DSAC).

La DTI est structurée en domaines qui sont chacun en charge de plusieurs pôles de compétences :
\begin{itemize}
\item Recherche \& développement, R et D
\item Exigences opérationnelles des systèmes, EOS
\item Gestion du trafic aérien, ATM
\item Communication, navigation, surveillance, CNS
\item Déploiement et Support Opérationnel, DSO
\end{itemize}
            \paragraph{}
Chaque pôles qui couvre un ensemble de fonctions et d’expertises.
Pole ATM/VIG :
Le pôle « Vol et information générale » (VIG) est responsable de la maîtrise d’ouvrage systèmes de traitement des plans de vol et informations générales, à ce titre, le pôle assure le suivi industriel de leur réalisation ou de leur acquisition. Le pôle VIG est également chargé de leur maintien en condition opérationnelles lorsqu’ils sont déployés.
Le pôle ATM/VIG est notamment responsable de la maîtrise d’ouvrage de systèmes déployés en outre-mer. L’aéroport de Tahiti (Polynésie française) a récemment été modernisé avec un système entièrement acquis auprès d’un industriel, couplé à un radar dans le cadre du projet TIARE, qui s’est terminé en 2009.

    \subsubsection{Les partenaires}
        Tissu local, sous-traitant.
        
        Relations au plan local, nationnal...

    \subsection{Situation géographique}
    
        Son emplacement.

\section[Controle aérien à Thaiti]{Le systeme de contrôle aérien mis en place à Tahiti}
    \subsection{Le système TIARE}
TIARE est un système de gestion du trafic aérien pour le centre de contrôle de Tahiti, en remplacement des systèmes vieillissants de visualisation du trafic (VIVO) et de gestion de plans de vol et d’informations générales (SIGMA). La superficie de l'espace aérien géré par le centre de contrôle de Tahiti s’étend sur 12 500 000 km2. Les situations de contrôle auxquelles doivent face les contrôleurs sont multiples, il y en a en effet à traiter les spécificités du contrôle océanique, du contrôle d’approche et inter-iles. Le système TIARE est construit à partir de plusieurs « produits sur étagère » :
\begin{itemize}
\item EUROCAT-X, système en charge du traitement radar et de la gestion plans de vols.
\item ATALIS, système en charge de la préparation des vols, de la gestion des NOTAM, et de la présentation d’informations générales au contrôleur tour et approche.
\end{itemize}

Les systèmes EUROCAT-X et ATALIS sont connectés au commutateur CAGOU, raccordé aux liaisons externes (RSFTA). ATALIS reçoit également des informations météorologiques en provenance du système local d’acquisition de ces données appelé CAOBS. EUROCAT-X est raccordé au radar secondaire du mont Marau et au réseau ACARS.

    \subsection{La zone \textsc{Aci}:\label{Aci}}
Une fonction de contrôle spécifique, nommée \textsc{Aci}\footnote{\textsc{Aci}: Area Common Interest, soit une zonne d'intérets commun} ou zone \textsc{Aci}, a été développée dans le système EUROCAT-X pour répondre à des besoins de contrôle. Il s’agit d’une zone particulière limitrophe à la \textsc{Fir}\footnote{\label{FIR} la \textsc{Fir} est la zone dans laquelle les contrôleurs doivent assurer le contrôle des vols} de Tahiti , dons la limite se situe à 50 miles nautiques de la \textsc{Fir}. La zone \textsc{Aci} encercle la \textsc{Fir}. Il est à noter que cette zone n’est pas sous la responsabilité des contrôleurs aériens français, cependant, les vols pénétrant dans cette région sont visualisés par le système Eurocat-X 

Ainsi en visualisant le trafic aérien dans la zone \textsc{Aci}, les contrôleurs peuvent maintenir les séparations entre les aéronefs. C'est-à-dire vérifier que les vols qui sont à l’extérieur et longent la \textsc{Fir} de Tahiti sont séparés des vols évoluant dans cette \textsc{Fir}.
