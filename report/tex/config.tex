%%%%%%%%%%%%%%%%%%%%%%%%%%%%%%%%%%%%%%%%%%%%%%%%%%%%%%%%%%%%%%%%%%%%%%%%%%%%%%%
%% Si vous voulez plus petit passez en 11pt voire encore plus petit en 10pt
\documentclass[a4paper,12pt,francais]{report}

%% Je suis francophone !
\usepackage[francais]{babel}
\usepackage[utf8]{inputenc}
\usepackage[T1]{fontenc}
\selectlanguage{french}

%% J'ai besoin des paquet:
\usepackage{longtable,geometry,setspace}
\usepackage[francais]{layout}
\usepackage{listings}
\usepackage{lastpage}
\usepackage[francais]{varioref}

%% J'aime bien pouvoir contrôler mes hauts de page !
\usepackage{fancyheadings}
\usepackage{fancyhdr}
\pagestyle{fancyplain}

%% Je veux pouvoir inclure des figures...
%% À commenter si vous voulez faire du DVI :
%
\usepackage[pdftex]{graphicx}

%% Je veux créer des Hyperdocuments
%% À commenter si vous voulez faire du DVI :
%  [pdftex,colorlinks=true,linkcolor=blue,citecolor=blue,urlcolor=blue]
%\usepackage{hyperref}
\ifpdf
    \pdfoutput=1
    \pdfcompresslevel=9
    \usepackage[pdftex=true,
    hyperindex=true,
    colorlinks=true,
    pdfpagelabels,
    linkcolor=blue,
    citecolor=red,
    urlcolor=blue,
    anchorcolor=red,
    plainpages=false,
    bookmarks=true]{hyperref}
    % réglage des paramètres pour la création des pdf
    % à mettre dans l'entête du document
    \def\docpdf{
    \hypersetup{
      pdftitle={\@typerapport},
      pdfauthor={\@auteur},
      pdfsubject={\@titre},
    }}
\else
    \usepackage[hypertex=true,
    hyperindex=true,
    colorlinks=false]{hyperref}
\fi

%%%%%%%%%%%%%%%%%%%%%%%%%%%%%%%%%%%%%%%%%%%%%%%%%%%%%%%%%%%%%%%%%%%%%%%%%%%%%%%
%% Je peux définir mes propres « commandes »...

%% pour citer une figure :
\newcommand{\figref}[1]{(cf.~Fig.~\vref{#1})}

%% pour citer une équation :
\newcommand{\nref}[1]{(cf.~\vref{#1})}

%% pour mettre l'emphase sur un mot ou un groupe de mots :
\newcommand{\empha}[1]{\textit{\textbf{#1}}}

% Je regele les marges pour appliquer le standard EIGSI
\geometry{%
  a4paper,
  body={150mm,240mm},
  left=40mm,top=30mm,
  headheight=10mm,headsep=7mm,
  marginparsep=0mm,
  marginparwidth=0mm}
%% Je definie les numérotation   
\setcounter{secnumdepth}{3}

%% Voilà des hauts de page comme je les aime :
\fancyhf{} %initialise tous les champs dans head et foot
\chead{\small Representation du trafic aérien de Tahiti dans Google Earth}
\lhead{\thesection}
\rhead{\small DGAC}
\lfoot{\small Satge été 2010}
\rfoot{\small Page: \thepage\ sur \pageref{LastPage}}
\cfoot{M. \textsc{Kervizic} Emmanuel - \textsc{Promo}2011}
\renewcommand{\headrulewidth}{1pt}% met la largeur du trait à 1.5pt
\renewcommand{\footrulewidth}{1pt}

\fancypagestyle{plain}{%
\renewcommand{\headrulewidth}{1pt}% met la largeur du trait à 1.5pt
\renewcommand{\footrulewidth}{1pt}}

%% Voilà mes légendes de figures comme je les aime
\makeatletter
\def\figurename{{\protect\sc \protect\small\bfseries Fig.}}
\def\f@ffrench{\protect\figurename\space{\protect\small\bf \thefigure}\space}
\let\fnum@figure\f@ffrench%
\let\captionORI\caption
\def\caption#1{\captionORI{\rm\small #1}}
\makeatother

%% Je met en forme le code python importé
\lstset{ %
language=Python,                % choose the language of the code
basicstyle=\scriptsize,         % the size of the fonts that are used for the code
numbers=left,                   % where to put the line-numbers
numberstyle=\scriptsize,        % the size of the fonts that are used for the line-numbers
stepnumber=1,                   % the step between two line-numbers. If it's 1 each line 
                                % will be numbered
numbersep=5pt,                  % how far the line-numbers are from the code
backgroundcolor=\color{white},  % choose the background color. You must add \usepackage{color}
showspaces=false,               % show spaces adding particular underscores
showstringspaces=false,         % underline spaces within strings
showtabs=false,                 % show tabs within strings adding particular underscores
frame=single,	                % adds a frame around the code
tabsize=2,	                    % sets default tabsize to 2 spaces
captionpos=b,                   % sets the caption-position to bottom
breaklines=true,                % sets automatic line breaking
breakatwhitespace=false,        % sets if automatic breaks should only happen at whitespace
title=\lstname,                 % show the filename of files included with \lstinputlisting;
                                % also try caption instead of title
escapeinside={\%*}{*)},         % if you want to add a comment within your code
morekeywords={*,self},            % if you want to add more keywords to the set
keywordstyle=\bf \color {blue},
%identifierstyle=\underline,
commentstyle=\color[gray]{0.5},
stringstyle=\color{red}
}


%%%%%%%%%%%%%%%%%%%%%%%%%%%%%%%%%%%%%%%%%%%%%%%%%%%%%%%%%%%%%%%%%%% 
% Définitions des variables et des macros pour les modifier pour la 
% page de titre. dans le texte, utiliser \wxxx pour ecrire xxx dans 
% le document, par exple : \wauteur ecrit l'auteur ...
% Période du stage
\def\@periode{Non renseigné~!}
\def\periode#1{\def\@periode{#1}}
\newcommand{\wperiode}{\@periode}
% Auteur du rapport
\def\@auteur{Non renseigné~!}
\def\auteur#1{\def\@auteur{#1}}
\newcommand{\wauteur}{\@auteur}
% Type de rapport (tfe, stage 2A, 3A, \ldots)
\def\@typerapport{Non renseigné~!}
\def\typerapport#1{\def\@typerapport{#1}}
\newcommand{\wtyperapport}{\@typerapport}
% Titre
\def\@titre{Non renseigné~!}
\def\titre#1{\def\@titre{#1}}
\newcommand{\wtitre}{\@titre}
% Tuteur entreprise
\def\@tuteurent{Non renseigné~!}
\def\tuteurent#1{\def\@tuteurent{#1}}
% Tuteur école
\def\@tuteurstage{Non renseigné~!}
\def\tuteurstage#1{\def\@tuteurstage{#1}}
% Logo entreprise (exple : logo_emac.png)
\def\@logoentrep{}
\def\logoentrep#1{\def\@logoentrep{#1}}
% Nom entreprise
\def\@nomentrep{Non renseigné~!}
\def\nomentrep#1{\def\@nomentrep{#1}}
% Adresse entreprise
\def\@addrentrep{Non renseigné~!}
\def\addrentrep#1{\def\@addrentrep{#1}}
% Logo de l'école, université, \ldots
\def\@logoecole{}
\def\logoecole#1{\def\@logoecole{#1}}
% taille des logos (width)
\newlength{\widthecole}
\newlength{\widthent}
\setlength{\widthecole}{0cm}
\setlength{\widthent}{0cm}
% macros utilisateurs pour spécifier les tailles
\newcommand{\tailleecole}[1]{\addtolength{\widthecole}{#1}}
\newcommand{\tailleent}[1]{\addtolength{\widthent}{#1}}

%% Je met en forme les paragraphe
%\onehalfspacing
%\singlespacing
%\setstretch{1.3}
\addtolength{\parskip}{0.5\baselineskip}
\setlength{\parindent}{0pt}
\setlength{\medskipamount}{\parskip}

%% Et les chapitre
\makeatletter
\def\@makechapterhead#1{%
  {\parindent \z@ \raggedright \normalfont
    \interlinepenalty\@M
   \LARGE \bfseries\thechapter\quad#1\par\nobreak
   \vskip 10\p@
  }}
\makeatother


%%%%%%%%%%%%%%%%%%%%%%%%%%%%%%%%%%%%%%%%%%%%%%%%%%%%%%%%%
%%Macros
%%%%%%%%%%%%%%%%%%%%%%%%%%%%%%%%%%%%%%%%%%%%%%%%%%%%%%%%
%% réglage des paramètres pour la création des pdf
%% à mettre dans l'entête du document
%\def\docpdf{
%\hypersetup{
  %pdftitle={\@typerapport},
  %pdfauthor={\@auteur},
  %pdfsubject={\@titre},
%}}
%% \myref produit une référence à un document numéroté
%% par son numéro et la page. #1 est le type de document, 
%% #2 est le label du document
%\newcommand{\myref}[1]{\ref{#1}, page~\pageref{#1}}
%% nouvelle commande : \chapitre pour prendre le fancy style en compte
%\newcommand{\chapitre}[1]{\chapter{#1} \thispagestyle{chapter}}
%% Création de la page de titre
%\newcommand{\pagedetitre}{
%\thispagestyle{empty}
%\begin{titlepage} %%% page de titre
%\begin{center}
%\includegraphics[width=\widthecole]{\@logoecole}\\
%\vspace{5mm}
%\rule{\linewidth}{0.8mm}\vspace{3mm}\\
%\setlength{\baselineskip}{1.5\baselineskip}
%{\Huge \textsc{\@titre}\par} \\
%\setlength{\baselineskip}{0.67\baselineskip}
%\vspace{3mm}
%{\rule{\linewidth}{0.8mm}}\\
%\vspace{1.2cm}
%\begin{tabularx}{\linewidth}{>{\centering}p{\linewidth}}
%{\LARGE \bf \@typerapport} \cr
%~~ \cr
%{\Large  \@nomentrep}\cr
%\end{tabularx}\\
%\vspace{1.3cm}
%\begin{tabularx}{\linewidth}{>{\raggedright}p{7.5cm} >{\raggedleft}p{4cm}}
%~~ & 
%\multirow{5}{0.5cm}{\includegraphics[width=\widthent]{\@logoentrep}}
%\tabularnewline 
%{\Large Tuteur école~: } \tabularnewline
%\hspace{8mm}{\Large \@tuteurstage }\tabularnewline
%~~ \tabularnewline
%{\Large Tuteur entreprise~: }\tabularnewline
%\hspace{8mm}{\Large \@tuteurent}\tabularnewline
%\end{tabularx}

%\vspace{3cm}

%{\Large \@auteur}\\
%\vspace{7mm}
%\large{---~\@periode~---}
%\end{center}
%\end{titlepage}
%}
%% Création de la page de titre détaillée (interieur)
%\newcommand{\pagetitredetail}{
%\thispagestyle{empty}
%\begin{titlepage} %%% page de titre
%\begin{center}
%\includegraphics[width=\widthecole]{\@logoecole}\\
%\vspace{3mm}
%\rule{\linewidth}{0.8mm}\vspace{3mm}\\
%\setlength{\baselineskip}{1.5\baselineskip}
%{\Huge \textsc{\@titre}\par} \\
%\setlength{\baselineskip}{0.67\baselineskip}
%\vspace{3mm}
%{\rule{\linewidth}{0.8mm}}\\
%\vspace{1cm}
%\begin{tabularx}{\linewidth}{>{\centering}p{\linewidth}}
%{\LARGE \bf \@typerapport} \tabularnewline
%~~ \tabularnewline
%{\Large  \@nomentrep}\tabularnewline
%\end{tabularx}\\
%\vspace{1cm}
%\begin{tabularx}{\linewidth}{>{\raggedright}p{7.5cm} >{\raggedleft}p{4cm}}
%~~ & 
%\multirow{5}{0.5cm}{\includegraphics[width=.8\widthent]{\@logoentrep}}
%\tabularnewline 
%{\Large Tuteur école~: } \tabularnewline
%\hspace{8mm}{\Large \@tuteurstage }\tabularnewline
%~~ \tabularnewline
%{\Large Tuteur entreprise~: }\tabularnewline
%\hspace{8mm}{\Large \@tuteurent}\tabularnewline
%\end{tabularx}\\
%\vspace{1cm}
%\begin{flushright}
%\@addrentrep
%\end{flushright}
%\vspace{8mm}

%{\Large \@auteur}\\
%\vspace{5mm}
%\large{---~\@periode~---}
%\end{center}
%\end{titlepage}
%}
%%%%%%%%%%%%%%%%%%%%%%%%%%%%%%%%%%%%%%%%%%%%%%%%%%%%%%%%%%%%%%%%%%%%%%%%
%%nouvelles commandes pour inserer des images sur les titres de parties. 
%% usage : part[nom court]{nom long}{ce qu'on veut}
%%                                   ^^^^^^^^^^^^^ (image, texte, ...)
%\makeatletter
%\def\@part[#1]#2#3{%
  %\ifnum \c@secnumdepth >-2\relax \refstepcounter {part}%
    %\addcontentsline{toc}{part}{\thepart \hspace {1em}#1}%
  %\else
    %\addcontentsline {toc}{part}{#1}%
  %\fi
  %\markboth {}{}{%
  %\centering \vfill
  %\interlinepenalty \@M \normalfont 
  %\ifnum \c@secnumdepth >-2\relax
    %\Huge \bfseries \partname ~\thepart \par \vskip 20\p@ 
  %\fi 
  %\Huge \bfseries #2
  %\par}%
  %{\centering
  %\vfill #3 \vfill}%
  %\@endpart}
%\makeatother

%%%%%%%%%%%%%%%%%%%%%%%%%%%%%%%%%%%%%%%%%%%%%%%%%%%%%%%%%%%%%%%%%%





%% Définitions des variables et des macros pour les modifier pour la 
%% page de titre. dans le texte, utiliser \wxxx pour ecrire xxx dans 
%% le document, par exple : \wauteur ecrit l'auteur ...
%% Période du stage
%\def\@periode{Non renseigné~!}
%\def\periode#1{\def\@periode{#1}}
%\newcommand{\wperiode}{\@periode}
%% Auteur du rapport
%\def\@auteur{Non renseigné~!}
%\def\auteur#1{\def\@auteur{#1}}
%\newcommand{\wauteur}{\@auteur}
%% Type de rapport (tfe, stage 2A, 3A, \ldots)
%\def\@typerapport{Non renseigné~!}
%\def\typerapport#1{\def\@typerapport{#1}}
%\newcommand{\wtyperapport}{\@typerapport}
%% Titre
%\def\@titre{Non renseigné~!}
%\def\titre#1{\def\@titre{#1}}
%\newcommand{\wtitre}{\@titre}
%% Tuteur entreprise
%\def\@tuteurent{Non renseigné~!}
%\def\tuteurent#1{\def\@tuteurent{#1}}
%% Tuteur école
%\def\@tuteurstage{Non renseigné~!}
%\def\tuteurstage#1{\def\@tuteurstage{#1}}
%% Logo entreprise (exple : logo_emac.png)
%\def\@logoentrep{}
%\def\logoentrep#1{\def\@logoentrep{#1}}
%% Nom entreprise
%\def\@nomentrep{Non renseigné~!}
%\def\nomentrep#1{\def\@nomentrep{#1}}
%% Adresse entreprise
%\def\@addrentrep{Non renseigné~!}
%\def\addrentrep#1{\def\@addrentrep{#1}}
%% Logo de l'école, université, \ldots
%\def\@logoecole{}
%\def\logoecole#1{\def\@logoecole{#1}}
%% taille des logos (width)
%\newlength{\widthecole}
%\newlength{\widthent}
%\setlength{\widthecole}{0cm}
%\setlength{\widthent}{0cm}
%% macros utilisateurs pour spécifier les tailles
%\newcommand{\tailleecole}[1]{\addtolength{\widthecole}{#1}}
%\newcommand{\tailleent}[1]{\addtolength{\widthent}{#1}}

%%%%%%%%%%%%%%%%%%%%%%%%%%%%%%%%%%%%%%%%%%%%%%%%%%%%%%%%%
%%Macros
%%%%%%%%%%%%%%%%%%%%%%%%%%%%%%%%%%%%%%%%%%%%%%%%%%%%%%%%

%% \myref produit une référence à un document numéroté
%% par son numéro et la page. #1 est le type de document, 
%% #2 est le label du document
%\newcommand{\myref}[1]{\ref{#1}, page~\pageref{#1}}
%% nouvelle commande : \chapitre pour prendre le fancy style en compte
%\newcommand{\chapitre}[1]{\chapter{#1} \thispagestyle{chapter}}
%% Création de la page de titre
%\newcommand{\pagedetitre}{
%\thispagestyle{empty}
%\begin{titlepage} %%% page de titre
%\begin{center}
%\includegraphics[width=\widthecole]{\@logoecole}\\
%\vspace{5mm}
%\rule{\linewidth}{0.8mm}\vspace{3mm}\\
%\setlength{\baselineskip}{1.5\baselineskip}
%{\Huge \textsc{\@titre}\par} \\
%\setlength{\baselineskip}{0.67\baselineskip}
%\vspace{3mm}
%{\rule{\linewidth}{0.8mm}}\\
%\vspace{1.2cm}
%\begin{tabularx}{\linewidth}{>{\centering}p{\linewidth}}
%{\LARGE \bf \@typerapport} \cr
%~~ \cr
%{\Large  \@nomentrep}\cr
%\end{tabularx}\\
%\vspace{1.3cm}
%\begin{tabularx}{\linewidth}{>{\raggedright}p{7.5cm} >{\raggedleft}p{4cm}}
%~~ & 
%\multirow{5}{0.5cm}{\includegraphics[width=\widthent]{\@logoentrep}}
%\tabularnewline 
%{\Large Tuteur école~: } \tabularnewline
%\hspace{8mm}{\Large \@tuteurstage }\tabularnewline
%~~ \tabularnewline
%{\Large Tuteur entreprise~: }\tabularnewline
%\hspace{8mm}{\Large \@tuteurent}\tabularnewline
%\end{tabularx}

%\vspace{3cm}

%{\Large \@auteur}\\
%\vspace{7mm}
%\large{---~\@periode~---}
%\end{center}
%\end{titlepage}
%}
%% Création de la page de titre détaillée (interieur)
%\newcommand{\pagetitredetail}{
%\thispagestyle{empty}
%\begin{titlepage} %%% page de titre
%\begin{center}
%\includegraphics[width=\widthecole]{\@logoecole}\\
%\vspace{3mm}
%\rule{\linewidth}{0.8mm}\vspace{3mm}\\
%\setlength{\baselineskip}{1.5\baselineskip}
%{\Huge \textsc{\@titre}\par} \\
%\setlength{\baselineskip}{0.67\baselineskip}
%\vspace{3mm}
%{\rule{\linewidth}{0.8mm}}\\
%\vspace{1cm}
%\begin{tabularx}{\linewidth}{>{\centering}p{\linewidth}}
%{\LARGE \bf \@typerapport} \tabularnewline
%~~ \tabularnewline
%{\Large  \@nomentrep}\tabularnewline
%\end{tabularx}\\
%\vspace{1cm}
%\begin{tabularx}{\linewidth}{>{\raggedright}p{7.5cm} >{\raggedleft}p{4cm}}
%~~ & 
%\multirow{5}{0.5cm}{\includegraphics[width=.8\widthent]{\@logoentrep}}
%\tabularnewline 
%{\Large Tuteur école~: } \tabularnewline
%\hspace{8mm}{\Large \@tuteurstage }\tabularnewline
%~~ \tabularnewline
%{\Large Tuteur entreprise~: }\tabularnewline
%\hspace{8mm}{\Large \@tuteurent}\tabularnewline
%\end{tabularx}\\
%\vspace{1cm}
%\begin{flushright}
%\@addrentrep
%\end{flushright}
%\vspace{8mm}

%{\Large \@auteur}\\
%\vspace{5mm}
%\large{---~\@periode~---}
%\end{center}
%\end{titlepage}
%}



