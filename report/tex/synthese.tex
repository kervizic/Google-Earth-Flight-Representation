%Méthode employée à consommateur de personne à disposition, produit très riche si compétence,  adapté et performant, Evolution désordonnée si pas maitrisé (base de données en plus ), changement des spécifications  en cours de projet, difficulté de rédaction de spécification produit fini concentre sur le dev et moins sur la doc.
%Pas de rédaction de manuel d’utilisateur,

\section{La gestion de projet}
L'utilisation de l'extreme programming pour gérer le projet aura été réellement bénéfique. On notera tout de même que cette méthode requière des clients extrêmement compétents et réactifs. En effet sans compétence de la part du client le projet peu rapidement tourner en rond.

Le fait de renouveler sans cesse les besoins et spécifications permet de réaliser un produit riche, adapté et performant. L'évolution quant à elle demande une maîtrise bien plus stricte qu'avec une méthode de gestion de projet plus classique sous peine de devenir rapidement désordonnée.

Nous avons utilisé la méthode agile pour préciser les besoins et obtenir une spécification applicable à la concrétisation d'un vrai logiciel. Ainsi nous pouvons aussi dire que nous sommes restés au niveau 1 d'un cycle en V

\section{Le projet}
Nous avons atteint un grand nombre d'objectif avec ce projet, il nous est capable d'analyser des plans de vol et de les mettre en corrélation avec les reports reçu de l'avion par l'intermédiaire de liaison satellite.

Le projet n'est pas fini. Un grand nombre d'améliorations restent à implémenter tels qu'une interface graphique ou encore une base de donnée \nref{evolution}.

\section{Un stage formateur}
    \subsection{Un apport technique}
J'ai pu au cours de ce stage approfondir et mettre en pratique mes connaissances en programmation python. Mais j'ai surtout pu découvrir le monde de l'aéronautique. 

En effet ce stage ma permis de voir les technologies utilisées dans les zones de contrôles. J'ai donc pu prendre connaissance des technologies de détection des avions de dernière génération (des fois pas encore mis en place) tels que les radars secondaires Mode S. Ou encore les systèmes de détections de collisions. J'ai également pu m'instruire sur leurs solutions de télécommunications (Satellite, \textsc{Vhf}) qui ne sont pas enseignées sous cet angle (pratique et non théorique) dans le cadre de mon cursus.

J'ai aussi pu découvrir le fonctionnement d'un serveur \textsc{Ntp}, les principe de la para-virtualisation ou encore la mise en place de réseaux privé virtuel. Autant de domaines n'étant pas en relation direct avec le stage mais qui auront une grande utilité dans mon avenir professionnel. 

Un autre point découvert dans la pratique aura été des méthodes de gestion de projet. La première étant le cycle en V du fait que ce soit celle qui est appliqué au sein de l'entreprise. La deuxième étant l'extreme programing utilisée pour le projet du stage.

L'ouverture d'esprit du personnel faisant partie ou travaillant en sous-traitance pour la \textsc{Dgac} y a fortement contribué. 

    \subsection{Des rapport humains}
Ce stage m'a également permis, notamment lors des pauses cafés ou repas du midi, d'échanger avec un grand nombre d'ingénieurs travaillant pour la \textsc{Dgac} ou en sous-traitance. 

Ce sont grâce à ces échanges que j'ai pu acquérir une grande partie des connaissances en aéronautique et en informatique citées précédemment. En effet ces personnes n'ont pas héritées à prendre un peu de leur temps pour me faire des schémas sur le fonctionnement des différentes technologies de radars ou encore retrouver leur rapport de test (benchmark) réalisé sur différentes mise en place de virtualisations d'OS (Xen, Kvm, VMWere). 

\section{Conclusion}
Ce stage aura été une expérience professionnel très enrichissante. Il m'aura permis de découvrir le monde de l'aéronautique. Il m'aura aussi permis de découvrir plusieurs méthodes de gestion de projet ainsi que des connaissances techniques variées.

Au delà des aspects pédagogiques techniques, il m'aura aussi permis de me familiariser avec le monde de l'entreprise du point de vue d'un ingénieur. 

Le contexte m'aura aussi sensibilisé sur la qualité. En effet dans l'aéronautique la gestion de la qualité est des plus importante du fait qu'un système ne peut pas tomber en panne sous peine d'avoir de grave conséquences. 
 









