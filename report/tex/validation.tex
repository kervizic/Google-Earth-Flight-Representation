%Tests et validation de la réalisation
%            Démarche pour tester le produit (manque pas des vols)
%            Un fichier même vol, mais fichiers avec des vols supplémentaires
%            Présentation du rendu
%Améliorations continues : à partir des tests, je repars dans le chapitre précédent (réalisation technique + nouveaux besoins (comparaison FPL//ADSC) ou correction)

\section{Les tests}
Nous avons effectué au cours de ce projet deux types de tests: les tests unitaires et les tests globaux.

    \subsection{Les tests unitaires}
En programmation informatique, le test unitaire est un procédé permettant de s'assurer du fonctionnement correct d'une partie déterminée d'un logiciel ou d'une portion d'un programme (appelée «unité» ou «module»).

On écrit un test pour confronter une réalisation à sa spécification. Le test définit un critère d’arrêt (état ou sorties à l’issue de l’exécution) et permet de statuer sur le succès ou sur l’échec d’une vérification. Grâce à la spécification, on est en mesure de faire correspondre un état d’entrée donné à un résultat ou à une sortie. Le test permet de vérifier que la relation d’entrée - sortie donnée par la spécification est bel et bien réalisée.

Petit rappel de définitions :
\begin{description}
\item[Test:] il s'agit d'une vérification par exécution.
\item[Vérification:] ce terme est utilisé dans le sens de contrôle d'une partie du logiciel. (Une «unité» peut ici être vue comme «le plus petit élément de spécification à vérifier»)
\end{description}

Il s'agit pour nous de tester un module, indépendamment du reste du programme, ceci afin de s'assurer qu'il répond aux spécifications fonctionnelles et qu'il fonctionne correctement en toutes circonstances.

Mais ces tests ne sufisent pas car il ne donnent pas assez de recule pour visualiser si l'ensemble du programme est fonctionnel. Il nous donne seulement une confirmation théorique.

    \subsection{Les tests globaux}
Comme nous l'avons cité précédement des test unitaires ne sont pas sufisant dans notre cas. En effet se projet etant un réél prototype dans le sens ou rien n'a été effectué de semblable auparavant, les spécification reste parfoit mal déterminée. On pourra cité comme exemple la scusture des message \textsc{Fpl} qui sont cencé avoir toujours la meme forme, mais qui se retrouve souvant avec des erreur du à la qualité de transmission du message.

Ces tests aurons donc pour but de valider le fait que toutes les parties développées indépendamment fonctionnent bien ensemble de façon cohérente.

Pour ce faire nous avons passé un grands nombre de fichier source à «parsé» Ce qui nous a permis de découvrire tout au long du projet un certains nombre d'erreur comme le fait de prendre en compte le nom de l'avion, aéroport de départ et heure de depart comme identifiant, celui-ci pouvant être le meme sur plusieur jours pour les vols cylciques. Dans ce cas les tests globaux nous on permis de decouvrir l'errur et nous a permis d'ajouter le jour de depart de l'avion dans l'identifiant afin que chaque identifiant reste bien unique.

\section{La validation}
Chaque fin de cycle de notre méthode de gestion de projet qui est l'Extreme Programing nous amene a une étape de validation. Celle ci conciste à verifier avec le client que le programme se comporte bien comme il le souhaitait.

Après chaque validation des spécification sont modifié car, bien que répondant à leur definition, elle ne réponde pas réelement au attente du client. D'autres spécifications sont crée et certaines annulées.

\section{Amélioration continue}
A partir des tests réalisés et de la validation avec le client comme cité précedement nous redéfinissons les besoins ainsi que les spécification. Cela nous amène donc à revenir au cycle des besoins \nref{besoins} dans notre méthode de gestion de projet \nref{extreme}.

Nous reprenons alors un cycle ce qui nous permetra d'amélioré le programme en continu.
