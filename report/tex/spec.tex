%- capture de fichiers
%- moulinette pour produire les KML
%- intégration dans googlearth

%Les besoins exprimé
\section{Spécifications obtenues}
Comme vous l'avez compris, utiliser une méthode agile comme l'extreme programming \nref{extreme} implique une perpétuelle réécriture des spécifications. Celle ci sont améliorée, réécrite, ajoutée tout au long du projet. C'est pourquoi dans ce rapport serra cité la dernière version des spécifications.

Les spécifications du logiciel sont les suivantes:
\begin{description}
    \item[Capture de fichiers de configuration:] Les points caractéristiques, route, zone de contrôle et \textsc{Aci}, doivent être récupérés dans les fichiers de configuration du système \textsc{Tiare} afin d'avoir la représentation la plus juste de ce que le système a. Ils doivent être gardé en mémoire pendent toute l'exécution du logiciel afin de pouvoir être utilisé. Les données seront enregistrées dans des objet le temps de l'exécution du programme afin de faciliter leur exploitation. La configuration du logiciel doit laisser à l'utilisateur la possibilité de spécifier le chemin du fichier de configuration.

    \item[Capture des données Plan de vol:] Les données Plan de vol doivent être récupérées dans les log du système \textsc{Tiare}. Par contre il doit être possible de les récupérer d'un autre fichier contenant des trames \textsc{Fpl} au format normalisé par la norme 4444 (cf Bib. \cite{4444}). Chaque plan de vol serra enregistré dans un objet ayant un identifiant comprenant: L'identifiant de l'avion, son aéroport de départ ainsi que l'heure et le jour de départ. Cet identifiant a pour but de les différentier et de les référencer dans le temps. La configuration du logiciel doit laisser à l'utilisateur la possibilité de spécifier le chemin du fichier de log.

    \item[Capture des données \textsc{Ads}:] Les données \textsc{Ads} doivent être récupérées dans les log du système \textsc{Tiare}. Il devra être aussi récupérer dans ces log les points de la position en fonction du temps calculé par le système entre deux report \textsc{Ads}. Les reports \textsc{Ads} et points calculé seront instancié par avion et par vol. L'identifiant de chaque vol sera donc composé de l'identifient de l'avion ainsi que de la date et l'heure du message de login. La configuration du logiciel doit laisser à l'utilisateur la possibilité de spécifier le chemin du fichier de log.

    \item[Les points caractéristiques:] Ces points devront être implémentés dans \textsc{Google Earth} avec la possibilité de les afficher ou non. La configuration du logiciel doit laisser à l'utilisateur la possibilité de spécifier la possibilité de rééditer ou non le fichier source \textsc{Google Earth}. Ces points serons représenté par un triangle de petite taille.

    \item[Les zones de Contrôle et \textsc{Aci}:] Les zones de contrôle et zones \textsc{Aci} devront être implémenté dans \textsc{Google Earth} avec la possibilité de les afficher ou non. La configuration du logiciel doit laisser à l'utilisateur la possibilité de spécifier la possibilité de rééditer ou non le fichier source \textsc{Google Earth}. Ces zones seront représentées par une surface colorée en 2 dimension.

    \item[Les routes:] Les routes devront être implémentées dans \textsc{Google Earth} avec la possibilité de les afficher ou non. La configuration du logiciel doit laisser à l'utilisateur la possibilité de spécifier la possibilité de rééditer ou non le fichier source \textsc{Google Earth}. C'est route seront représentée par une ligne de couleur Jaune. Les points définissant cette route ne seront pas illustrés afin de ne pas faire de doublon avec les points caractéristique. Les coordonnées des points de chaque route devront être défini à partir des points caractéristique en mémoire.

    \item[Les plan de vol:] Les plans de vol devront être implémenté dans \textsc{Google Earth} avec la possibilité de les afficher ou non. La configuration du logiciel doit laisser à l'utilisateur la possibilité de spécifier la possibilité de rééditer ou non le fichier source \textsc{Google Earth}. Les plans de vol doivent pouvoir être visualisé dans \textsc{Google Earth} en fonction du temps. Pour se faire une heure théorique de passage sera calculé par le programme pour chaque point définissant le plan de vol. Toutes les informations concernant chaque plan de vol tel que ça route, les points constituant sa route et sa situation dans le temps devront être regroupé dans un dossier. Le message FPL de l'avion doit être visible dans la description de ce dossier. Les plans de vol seront visible durant toute la durée du vol et représenté par une ligne noire. La visualisation dans le temps sera représentée par un segment de couleur choisie aléatoirement pour chaque vol défini par les deux points les plus proches de l'heure en paramètre dans le logiciel (un point avant et un point après). Ce segment et ses points ne serons visible qu'à partir de l'heure du premier points jusqu'à l'heure du deuxième. 

    \item[l'intersection du plan de vol avec la zone \textsc{Aci}:] L'intersection, si elle a lieu, entre le plan de vol et la zone \textsc{Aci} doit être calculé, définie et représenté dans \textsc{Google Earth} par un point rouge accompagné du nom de l'avion et de l'heure d'intersection affiché en rouge également. Ces points devront être contenu dans le dossier du concerné.

    \item[Les report \textsc{Ads}:] Chaque report \textsc{Ads} serra composé de ces points de report ainsi que des points calculé par le système. Chaque report serra regroupé dans un dossier par vol et aura comme description le message reçu. Chaque point calculé par le système sera attribué et regroupé avec le report précèdent. Les vols seront représenté par une ligne blanche retraçant tout les report reçu, ainsi que chaque point affiché dans le temps. L'intérêt étant de visualisé l'écart entre le chemin parcouru par l'avion et le plan de vol déposé ainsi que la différence entre la trajectoire de l'avion et celle calculée par le système \textsc{Tiare}. 
\end{description}
















