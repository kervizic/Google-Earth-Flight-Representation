%- capture de fichiers
%- moulinette pour produire les KML
%- intégration dans googlearth

%Les besoins exprimé

Comme vous l'avez compris, utiliser une méthode agile comme l'extreme programming \nref{xtreme} implique une perpétuelle réecriture des spécifications. Celle ci sont améliorée, réécrite, ajoutée tout au long du projet. C'est pourquoi dans ce rapport serra cité la derniere vertion des spécification.

les spécification du logitielle sont les suivante:
\begin{description}
    \item[Capture de fichiers de configuration] Les points charactéristiques, route, zone de controle et \textsc{Aci}, doivent être récupérés dans les fichiers de configuration du système \textsc{Tiare} afin d'avoir la representation la plus juste de ce que le système a. Il doivent etre gardé en memoire pendent toute l'execution du logiciel afin de pouvoir etre utilisé. Les donnée seront enregistrées dans des objet le temps de l'exection du programme afin de faciliter leur exploitation. La configuration du logiciel doit laisser a l'utilisateur la possibilité de spécifier le chemin du fichier de configuration.
    \item[Capture des données Plan de vol] Les données Plan de vol doivent etre récupérée dans les log du système \textsc{Tiare}. Par contre il doit être possible de les récuperer d'un autre fichier contenent des trame FPL au format normalisé par la norme 4444 (cf Bib. \cite{4444}). Chaque plan de vol serra enregistré dans un objet ayant un identifiant comprenant: L'identifiant de l'avion, son aéroport de depart ainsi que l'heure et le jour de depart. Cet identifiant a pour but de les diferentier et de les réferencer dans le temps. La configuration du logiciel doit laisser a l'utilisateur la possibilité de spécifier le chemin du fichier de log.
    \item[Capture des données \textsc{ADS}] Les données \textsc{ADS} doivent etre récupérée dans les log du système \textsc{Tiare}. Il devra être aussi récupérer dans ces log les points de la position en fonction du temps calculé par le système entre deux report \textsc{Ads}. Les reports \textsc{Ads} et points calculé seront instencié par avion et par vol. L'identifiant de chaque vol sera donc composé de l'identifient de l'avion ainsi que de la date et l'heure du message de login. La configuration du logiciel doit laisser a l'utilisateur la possibilité de spécifier le chemin du fichier de log.
    \item[Les points caractéristiques] Ces points devront être implémenté dans \textsc{Google Earth} avec la possibilité de les afficher ou non. La configuration du logiciel doit laisser a l'utilisateur la possibilité de spécifier la possibilité de réediter ou non le fichier source \textsc{Google Earth}. Ces point seron représenté par un triangle de petite taille.
    \item[Les zonnes de Contrôle et \textsc{Aci}] Les zonnes de controle et zonnes \textsc{Aci} devront être implémenté dans \textsc{Google Earth} avec la possibilité de les afficher ou non. La configuration du logiciel doit laisser a l'utilisateur la possibilité de spécifier la possibilité de réediter ou non le fichier source \textsc{Google Earth}. Ces zonnes seront représentée par une surface colorée en 2 dimention.
    \item[Les routes] Les routes devront être implémenté dans \textsc{Google Earth} avec la possibilité de les afficher ou non. La configuration du logiciel doit laisser a l'utilisateur la possibilité de spécifier la possibilité de réediter ou non le fichier source \textsc{Google Earth}. Cest route seront représentée par une ligne de couleur Jaune. Les points definissant cette route ne seront pas ilustré afin de ne pas faire de doublon avec les points caractéristique. Les coordonée des points de chaque route devront être defini a partir des points caractéristique en mémoire.
    \item[Les plan de vol] Les plans de vol devront être implémenté dans \textsc{Google Earth} avec la possibilité de les afficher ou non. La configuration du logiciel doit laisser a l'utilisateur la possibilité de spécifier la possibilité de réediter ou non le fichier source \textsc{Google Earth}. Les plan de vol doivent pouvoir être visualisé dans \textsc{Google Earth} en fonction du temps. Pour se faire une heure théorique de passage sera calculé par le programme pour chaque point definissant le plan de vol. Toutes les informations concernant chaque plan de vol tel que ca route, les points constituant sa route et sa situation dans le temps devront etre regroupé dans un dossier. Le message FPL de l'avion doit être visible dans la description de ce dossier. Les plans de vol seront visible durant toute la durée du vol et représenté par une ligne noire. La visualisation dans le temps sera représenté par un segment de couleur choisie aléatoirement pour chaque vol defini par les deux points les plus proche de l'heure en parametre dans le logiciel(un point avant et un point après). Ce segment et ses points ne serons visible qu'a partir de l'heure du premier points jusqu'a l'heure du deuxieme. 
    \item[l'itersection du plan de vol avec la zone \textsc{Aci}] L'intersetion, si elle a lieu, entre le plan de vol et la zone \textsc{Aci} doit être calculé, définie et representé dans \textsc{Google Earth} par un point rouge accompagné du nom de l'avion et de l'heure d'intersection affiché en rouge egalement. Ces points devront être contenu dans le dossier du concerné.
    \item[Les report \textsc{Ads}] Chaque report \textsc{Ads} serra composé de ces points de report ainsi que des points calculé par le système. Chaque report serra regroupé dans un dossier par vol et aura comme description le message recu. Chaque point calculé par le systeme sera attribué et regroupé avec le report précedent. Les vol sront representé par une ligne blanche retracant tout les report recu, ainsi que chaque point affiché dans le temps. L'interet etant de visualisé l'ecart entre le chemin parcouru par l'avion et le plan de vol déposé ainsi que la diferenec entre la trajectiore de l'avion et celle calculée par le système \textsc{Tiare}. 
\end{description}
















