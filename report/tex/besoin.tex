%Expression du besoin par les clients
%Enumère le besoin,
%micro analyse,
%Client dirigiste (python , google earth) mais sans besoin précis pour l’utilisation du produit.
%Risque que cela ne marche pas
 
\section{L'objectif initial du projet}
L’objectif principal est de pouvoir réaliser un logiciel banalisé et ergonomique permettant de
représenter l’ensemble des données de contrôle (repères, balises, secteurs...) afin de pouvoir
visualiser le trafic aérien circulant dans la \textsc{Fir} et la zone \textsc{Aci}.
Les bénéfices attendus de cet outil sont :
\begin{itemize}
\item l’amélioration de l’analyse et de la compréhension visuelle du trafic aérien de Tahiti,
\item la possibilité d’élaborer de statistiques à partir des fonctions de calcule du logiciel,
\item une aide dans le travail de définition des points d’entrée dans la zone \textsc{Aci} que réalise le
service de contrôle de Tahiti.
\end{itemize}

\section{Les besoins}
Au debut de projet les besoins initiaux ont été definis. Nous verons par la suite comment ceux-ci ont pu evoluer. Il faut noter que le client est assez dirigiste, il a deja vu ce produit pour d'autres applications et a donc une vue global de ce qu'il souhaite en sortie. A savoir:
\begin{itemize}
    \item Une application etant basée sur le logiciel \textsc{Google Earth}.
    \item Python comme langage de programmation
\end{itemize}
Par contre le besoins précis de l'utilisation du produit reste indeterminée. C'est pourquoi nous avons orienté notre gestion de projet vers une méthode dite agile (cf. \vref{xtreme}). Cette méthode nous permettra de redefinir les besoins tout au long du projet en fonction de ce qui a déja été réalisé. Et ainsi obtenir un produit correspondant au mieux a ce que le client aurait pu esperer.

Lors du lancement du projet les besoins étaient:
\begin{itemize}
    \item Représenter le trafic aérien déposé par les plans de vol dans la zone de contrôle de \textsc{Tahiti} dans \textsc{Google Erath}.
    \item Visualiser la configuration de la plateforme \textsc{Tiare} (zone de controle, point characteristique ...)
\end{itemize}
Tout au long du projet de nouveau besoins sont apparus tel que:
\begin{itemize}
    \item Representer le trafic aérien en fontion du temps
    \item Definir approximativement l'heure d'entrée de et sortie des avion dans la \textsc{Fir} (cf. \vref{Fir}) en fonction de leur plan de vol déposé.
    \item Visualiser le vol des avions en temps réel grace aux données \textsc{Ads} (cf. \vref{Ads}).
    \item Visualiser le positionement des avions estimer par le système \textsc{Teiare} entre deux reports \textsc{Ads} afin de visualiser l'iterpretation des données recue par le système.
    \item Diferentier les type de vol en quatre categories: Entrant, Sortant, Transit, Interne. 
\end{itemize}

\section{Les risques}
Lorsque l'on a comme projet de réaliser une appliquation qui a deja été réalisé par le passé nous avons une base sur laquel se referencer (en terme de methode, de temps, de couts). Hors sur un projet tel que le nôtre ou mème aucun prototype n'a encore été réalisé le risque que cela ne fonctionne pas est très elevé.

C'est pour cela qu'une methode de gestion de projet dite agile decrite ci-dessous (\vref{xtreme}) à été utilisé. Cette méthode nous a permis d'avancer petit a petit afin de suicité des besoins "réalisable". Contrairement à la methode en V utilisée originelement a la \textsc{Dti} ou les besoins et les spésification sont déterminé avant le bebut de la réalisation technique.
