Expression du besoin par les clients\par
Enumère le besoin ,\par
micro analyse ,\par
Client dirigiste (python , google earth) mais sans besoin précis pour l’utilisation du produit .\par
Risque que cela ne marche pas
 
\section{Les objectifs et besoins du projet}
    \subsection{L'objectif initial du projet}
L’objectif principal est de pouvoir réaliser un logiciel banalisé et ergonomique permettant de
représenter l’ensemble des données de contrôle (repères, balises, secteurs...) afin de pouvoir
visualiser le trafic aérien circulant dans la \textsc{Fir} et la zone \textsc{Aci}.
Les bénéfices attendus de cet outil sont :
\begin{itemize}
\item l’amélioration de l’analyse et de la compréhension visuelle du trafic aérien de Tahiti,
\item la possibilité d’élaborer de statistiques à partir des fonctions de calcule du logiciel,
\item une aide dans le travail de définition des points d’entrée dans la zone \textsc{Aci} que réalise le
service de contrôle de Tahiti.
\end{itemize}

    \subsection{Le besoin}
        \subsubsection{Le besoin au début du projet}
Le besoin était initialement de pouvoir visualiser les plans de vol des avions afin de pouvoir aider les contrôleurs à déterminer l'heure d'entrée approximative de l'avion dans la zone \textsc{Aci} (cf. \vref{Aci})
Le logiciel devant être portable et adaptable, l'utilisation langage Python a été défini comme l'une des meilleur alternative.
Tout au long du projet de nouveau besoin se sont fait ressentir. Ceux-ci comme nous pourrons le constater dans la suite du rapport nous a amener à modifier les objectifs du projet.

        \subsubsection{Un besoin redéfinit tout au long du projet}
On a très vite constaté, ce e dès les premières représentations,  que la création d'un logiciel qui pouvait représenter les données du système TIARE dans Google Eath pourrait avoir plusieurs intérêts:
\begin{itemize}
\item Visualiser comment le Système interprète les données
\item Comparer les plan de vol déposé avec les vols réalisé
\item Visualiser les zone de contrôles
\item Évaluer la mise en place du système \textsc{Ads}
\end{itemize}
C'est pourquoi les besoins on été redéfinit en cours de projet.

        \subsubsection{Des besoins redéfinis}
Les nouveaux besoins définis sont donc de pouvoir disposer d'une maquette (Appellée en anglais: Poc\footnote{Poc: Proof of concept, soit une démonstration de faisabilité}) qui serrait représentative de toutes les possibilités que pourrait apporter un logiciel de ce type.

Cette maquette doit être capable de:
\begin{itemize}
\item définir un point d'entré approximatif d'un plan de vol dans la zone \textsc{Aci}.
\item Visualiser les zone de contrôles
\item Visualiser tous les points caractéristique du système (Aéroport, balises …)
\item Visualiser les routes définie dans le système.
\item Visualiser les plan de vol en fonction du temps
\item Visualiser les report \textsc{Ads} ainsi que les points calculés par le système \textsc{Tiare}
\item Différencier les vols interne, sortant, entrant et en transit.
\end{itemize}

L’intérêt de cette maquette serrait de définir les spécification d'un logiciel qui pourrait être ensuite réalisé et mis en production dans des sites ou la sécurité n'est pas à négliger. Il aurait aussi l’intérêt de définir les fonctionnalités nécessaires à un utilisateur donné.
