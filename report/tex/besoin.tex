%Expression du besoin par les clients
%Énumère le besoin,
%micro analyse,
%Client dirigiste (python , google earth) mais sans besoin précis pour l’utilisation du produit.
%Risque que cela ne marche pas

\section{Expression du besoin} 
    \subsection{Les besoins initiaux}
L’objectif initial était de pouvoir réaliser un logiciel banalisé et ergonomique permettant de
représenter l’ensemble des données de contrôle (repères, balises, secteurs...) afin de pouvoir
visualiser le trafic aérien circulant dans la \textsc{Fir} et la zone \textsc{Aci}.
Les bénéfices attendus de cet outil sont :
\begin{itemize}
\item l’amélioration de l’analyse et de la compréhension visuelle du trafic aérien de Tahiti,
\item la possibilité d’élaborer de statistiques à partir des fonctions de calcule du logiciel,
\item une aide dans le travail de définition des points d’entrée dans la zone \textsc{Aci} que réalise le
service de contrôle de Tahiti.
\end{itemize}\medskip

    \subsection{L’évolution des besoins}
Au début de projet les besoins initiaux ont été définis. Nous verrons par la suite comment ceux-ci ont pu évoluer. Il faut noter que le client est assez dirigiste, il a déjà vu ce produit pour d'autres applications et a donc une vue global de ce qu'il souhaite en sortie. A savoir:
\begin{itemize}
    \item Une application étant basée sur le logiciel \textsc{Google Earth}.
    \item Python comme langage de programmation
\end{itemize}\medskip
L’objectif du choix de ces outils était aussi pour le client l’assurance de proposer à l’issue du stage une maquette complètement fonctionnelle. C'est-à-dire qu’il fallait déjouer la difficulté technique, comme représenter du trafic sur une sphère, pour se concentrer sur les besoins suscités par cet outil. En outre la durée du stage et la part consacrée à la rédaction du rapport de stage ne permettaient pas d’innover en créant un logiciel de toute pièce.

C'est pourquoi nous avons orienté notre gestion de projet vers une méthode dite agile \nref{extreme}. Cette méthode nous permettra de redéfinir les besoins tout au long du projet en fonction de ce qui a déjà été réalisé. Et ainsi obtenir un produit correspondant au mieux a ce que le client aurait pu espérer.

Lors du lancement du projet les besoins étaient:
\begin{itemize}
    \item Représenter le trafic aérien déposé par les plans de vol dans la zone de contrôle de \textsc{Tahiti} dans \textsc{Google Earth}.
    \item Visualiser la configuration de la plate-forme \textsc{Tiare} (zone de contrôle, point caractéristique ...)
\end{itemize}\medskip

Au fur et à mesure de la progression et des possibilités du logiciel, le client a affiné ses besoins et a rajouté les éléments suivants:
\begin{itemize}
    \item Représenter le trafic aérien en fonction du temps
    \item Définir approximativement l'heure d'entrée de et sortie des avion dans la \textsc{Fir} \nref{Fir} en fonction de leur plan de vol déposé.
    \item Visualiser le vol des avions en temps réel grâce aux données \textsc{Ads} \nref{Ads}.
    \item Visualiser le positionnement des avions estimé par le système \textsc{Tiare} entre deux reports \textsc{Ads} afin de visualiser l'interprétation des données reçue par le système.
    \item Différentier les type de vol en quatre catégories: Entrant, Sortant, Transit, Interne. 
\end{itemize}\medskip

    \subsection{Les risques}
Lorsque l'on a comme projet de réaliser une application qui a déjà été réalisé par le passé nous avons une base sur la-quel se référencer (en terme de méthode, de temps, de coûts). Hors sur un projet tel que le nôtre ou même aucun prototype n'a encore été réalisé le risque que cela ne fonctionne pas est très élevé.

C'est pour cela qu'une méthode de gestion de projet dite agile décrite ci-dessous \nref{extreme} à été utilisé. Cette méthode nous a permis d'avancer petit a petit afin de susciter des besoins "réalisable". 

Défaut de la méthode agile: besoins sans fin, nécessité d'avoir un groupe restreint de personne, nécessité d’avoir un expert pour guider.









