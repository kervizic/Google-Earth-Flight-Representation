\documentclass[a4paper,12pt]{report}
\usepackage{longtable,geometry}
\usepackage[francais]{babel}
\usepackage[francais]{layout}
\usepackage[utf8]{inputenc}
\usepackage[T1]{fontenc}
\usepackage{listings}
\usepackage{graphicx}
\usepackage{fancyhdr}
\usepackage{lastpage}
\usepackage[francais]{varioref}

\geometry{%
  a4paper,
  body={150mm,240mm},
  left=40mm,top=30mm,
  headheight=10mm,headsep=7mm,
  marginparsep=0mm,
  marginparwidth=0mm}

\lstset{ %
language=Python,                % choose the language of the code
basicstyle=\scriptsize,         % the size of the fonts that are used for the code
numbers=left,                   % where to put the line-numbers
numberstyle=\scriptsize,        % the size of the fonts that are used for the line-numbers
stepnumber=1,                   % the step between two line-numbers. If it's 1 each line 
                                % will be numbered
numbersep=5pt,                  % how far the line-numbers are from the code
backgroundcolor=\color{white},  % choose the background color. You must add \usepackage{color}
showspaces=false,               % show spaces adding particular underscores
showstringspaces=false,         % underline spaces within strings
showtabs=false,                 % show tabs within strings adding particular underscores
frame=single,	                % adds a frame around the code
tabsize=2,	                    % sets default tabsize to 2 spaces
captionpos=b,                   % sets the caption-position to bottom
breaklines=true,                % sets automatic line breaking
breakatwhitespace=false,        % sets if automatic breaks should only happen at whitespace
title=\lstname,                 % show the filename of files included with \lstinputlisting;
                                % also try caption instead of title
escapeinside={\%*}{*)},         % if you want to add a comment within your code
morekeywords={*,self},            % if you want to add more keywords to the set
keywordstyle=\bf \color {blue},
%identifierstyle=\underline,
commentstyle=\color[gray]{0.5},
stringstyle=\color{red}
}

\ifpdf
\usepackage[pdftex=true,
hyperindex=true,
colorlinks=true,
pdfpagelabels,
linkcolor=blue,
citecolor=red,
urlcolor=blue,
anchorcolor=red]{hyperref}
\else
\usepackage[hypertex=true,
hyperindex=true,
colorlinks=false]{hyperref}
\fi

\pdfcompresslevel=9

\fancyhf{}
\chead{\small Representation du trafic aérien de Tahiti dans Google Earth}
\lhead{\thesection}
\rhead{\small DGAC}
\lfoot{\small Satge été 2010}
\rfoot{\small Page: \thepage\ sur \pageref{LastPage}}
\cfoot{M. \textsc{Kervizic} Emmanuel - \textsc{Promo}2011}

\pagestyle{fancy}
%\fancyhf{} % on efface tout et on recommence
%% EN TÊTE :
%% initiales à droite sur page paire , à gauche sur page impaire :
%\fancyhead[R]{DGAC}
%% numéro de page au centre :
%%\fancyhead[R]{\thepage}
%\fancyhead[L]{Representation du trafic aérien de Tahiti dans Google Earth}
%% numéro de section à droite sur page impaire , à gauche sur page paire :
%%\fancyhead[LE,RO]{\thesection}
%% PIED DE PAGE :
%% une image à droite sur page impaire , à gauche sur page paire :
%%\fancyfoot[RO,LE]{\includegraphics[height=4ex]{punch}}
%\fancyfoot[L]{M. \textsc{Kervizic} Emmanuel / \textsc{Promo}2011}
%\fancyfoot[R]{\thepage\ sur \pageref{LastPage}}
%% titre à gauche sur page impaire , à droite sur page paire :
%%\fancyfoot[LO,RE]{%
%%Tout ce que vous avez toujours voulu savoir sur \LaTeX{}}
%% épaisseur des traits
\renewcommand{\headrulewidth}{1.5pt}
\renewcommand{\footrulewidth}{1.5pt}

