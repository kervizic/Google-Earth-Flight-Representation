% Configuration du fichier final
\documentclass[a4paper,12pt]{report}
\usepackage{longtable,geometry}
\usepackage[francais]{babel}
\usepackage[francais]{layout}
\usepackage[utf8]{inputenc}
\usepackage[T1]{fontenc}
\usepackage{listings}
\usepackage{graphicx}
\usepackage{fancyhdr}
\usepackage{lastpage}
\usepackage[francais]{varioref}

\geometry{%
  a4paper,
  body={150mm,240mm},
  left=40mm,top=30mm,
  headheight=10mm,headsep=7mm,
  marginparsep=0mm,
  marginparwidth=0mm}

\lstset{ %
language=Python,                % choose the language of the code
basicstyle=\scriptsize,         % the size of the fonts that are used for the code
numbers=left,                   % where to put the line-numbers
numberstyle=\scriptsize,        % the size of the fonts that are used for the line-numbers
stepnumber=1,                   % the step between two line-numbers. If it's 1 each line 
                                % will be numbered
numbersep=5pt,                  % how far the line-numbers are from the code
backgroundcolor=\color{white},  % choose the background color. You must add \usepackage{color}
showspaces=false,               % show spaces adding particular underscores
showstringspaces=false,         % underline spaces within strings
showtabs=false,                 % show tabs within strings adding particular underscores
frame=single,	                % adds a frame around the code
tabsize=2,	                    % sets default tabsize to 2 spaces
captionpos=b,                   % sets the caption-position to bottom
breaklines=true,                % sets automatic line breaking
breakatwhitespace=false,        % sets if automatic breaks should only happen at whitespace
title=\lstname,                 % show the filename of files included with \lstinputlisting;
                                % also try caption instead of title
escapeinside={\%*}{*)},         % if you want to add a comment within your code
morekeywords={*,self},            % if you want to add more keywords to the set
keywordstyle=\bf \color {blue},
%identifierstyle=\underline,
commentstyle=\color[gray]{0.5},
stringstyle=\color{red}
}

\ifpdf
\usepackage[pdftex=true,
hyperindex=true,
colorlinks=true,
pdfpagelabels,
linkcolor=blue,
citecolor=red,
urlcolor=blue,
anchorcolor=red]{hyperref}
\else
\usepackage[hypertex=true,
hyperindex=true,
colorlinks=false]{hyperref}
\fi

\pdfcompresslevel=9

\fancyhf{}
\chead{\small Representation du trafic aérien de Tahiti dans Google Earth}
\lhead{\thesection}
\rhead{\small DGAC}
\lfoot{\small Satge été 2010}
\rfoot{\small Page: \thepage\ sur \pageref{LastPage}}
\cfoot{M. \textsc{Kervizic} Emmanuel - \textsc{Promo}2011}

\pagestyle{fancy}
%\fancyhf{} % on efface tout et on recommence
%% EN TÊTE :
%% initiales à droite sur page paire , à gauche sur page impaire :
%\fancyhead[R]{DGAC}
%% numéro de page au centre :
%%\fancyhead[R]{\thepage}
%\fancyhead[L]{Representation du trafic aérien de Tahiti dans Google Earth}
%% numéro de section à droite sur page impaire , à gauche sur page paire :
%%\fancyhead[LE,RO]{\thesection}
%% PIED DE PAGE :
%% une image à droite sur page impaire , à gauche sur page paire :
%%\fancyfoot[RO,LE]{\includegraphics[height=4ex]{punch}}
%\fancyfoot[L]{M. \textsc{Kervizic} Emmanuel / \textsc{Promo}2011}
%\fancyfoot[R]{\thepage\ sur \pageref{LastPage}}
%% titre à gauche sur page impaire , à droite sur page paire :
%%\fancyfoot[LO,RE]{%
%%Tout ce que vous avez toujours voulu savoir sur \LaTeX{}}
%% épaisseur des traits
\renewcommand{\headrulewidth}{1.5pt}
\renewcommand{\footrulewidth}{1.5pt}


%¯¯¯¯¯¯¯¯¯¯¯¯¯

\date{10 septembre 2010}
\title{Rapport de stage Élève Ingénieur}
\author{Emmanuel \textsc{Kervizic}}

\begin{document}

% Table des matieres
\renewcommand{\baselinestretch}{1.1}\small \normalsize
\setcounter{tocdepth}{2}  
\tableofcontents
%% Je met en forme les paragraphe
\addtolength{\parskip}{0.5\baselineskip}
\setlength{\parindent}{0pt}
\setlength{\medskipamount}{\parskip}
\renewcommand{\baselinestretch}{1.2}\small \normalsize

% Debut des section
\chapter*{Introduction} %Omission de la numérotation
\addcontentsline{toc}{chapter}{Introduction} %Ajout tout de meme a la table des matieres
% Insertion de fichier
\section{Présentation de l'entreprise}


    \subsection{Description générale}
        \subsubsection{Activités de l'entreprise et historique}
            \paragraph{}
            Quelques chiffres

            \paragraph{}

        \subsubsection{localisation}
            \paragraph{}
            Ses domaines d'activités.
            \paragraph{}

        \subsubsection{Organisation}
            \paragraph{}
            Ses domaines d'activités.
            \paragraph{}

            Son pôle de recherche.
            \paragraph{}
            Ses compétences, son marché.
        \subsubsection{Présentation générale du service}
            \paragraph{}
La Direction des Services de la Navigation Aérienne est chargée de rendre le service de navigation aérienne pour l’Etat français. A ce titre, la DSNA est responsable de rendre les services de circulation aérienne, d’information aéronautique et d’alerte sur le territoire national et ceux d’outre-mer (DOM, TOM , POM). La DSNA s’appuie sur deux directions pour exécuter cette mission:
\begin{itemize}
\item La Direction des opérations ou DO,
\item la Direction de la Technique et de l’Innovation ou DTI.
\end{itemize}
La DO est l’acteur opérationnel du contrôle aérien tandis que la DTI est chargé du volet technique. Celui-ci consiste à réaliser ou acquérir les systèmes qui participent à l’exercice du contrôle aérien. Il s’agit de systèmes informatiques permettant d’assister le contrôleur dans ses activités, de chaines radios pour communiquer avec les aéronefs, de systèmes de traitement de l’information météorologique…
            \paragraph{}
La DTI réalise également de nombreuses études pour traiter les besoins des utilisateurs et les évolutions réglementaires. La DTI réalise le déploiement et le support opérationnel des systèmes qu’elle acquiert ou réalise. 
            \paragraph{}
Enfin la DTI fait viser ses systèmes, procédures et formation par l’autorité de surveillance nationale (Direction de la Sécurité de l'Aviation Civile ou DSAC).
            \paragraph{}
La DTI est structurée en domaines qui sont chacun en charge de plusieurs pôles de compétences :
\begin{itemize}
\item Recherche \& développement, R et D
\item Exigences opérationnelles des systèmes, EOS
\item Gestion du trafic aérien, ATM
\item Communication, navigation, surveillance, CNS
\item Déploiement et Support Opérationnel, DSO
\end{itemize}
            \paragraph{}
Chaque pôles qui couvre un ensemble de fonctions et d’expertises.
Pole ATM/VIG :
Le pôle « Vol et information générale » (VIG) est responsable de la maîtrise d’ouvrage systèmes de traitement des plans de vol et informations générales, à ce titre, le pôle assure le suivi industriel de leur réalisation ou de leur acquisition. Le pôle VIG est également chargé de leur maintien en condition opérationnelles lorsqu’ils sont déployés.
Le pôle ATM/VIG est notamment responsable de la maîtrise d’ouvrage de systèmes déployés en outre-mer. L’aéroport de Tahiti (Polynésie française) a récemment été modernisé avec un système entièrement acquis auprès d’un industriel, couplé à un radar dans le cadre du projet TIARE, qui s’est terminé en 2009.

    \subsubsection{Les partenaires}
        \paragraph{}
        Tissu local, sous-traitant.
        \paragraph{}
        Relations au plan local, nationnal...

    \subsection{Situation géographique}
        \paragraph{}
        Son emplacement.

    % ...

\newpage

\section{Environnement de travail}
% ...


\chapter{Contexte}
    %\subsection{Description générale}
        %\subsubsection{Activités de l'entreprise et historique}
            %\paragraph{}
            %Quelques chiffres

            %\paragraph{}

        %\subsubsection{localisation}
            %\paragraph{}
            %Ses domaines d'activités.
            %\paragraph{}

        %\subsubsection{Organisation}
            %\paragraph{}
            %Ses domaines d'activités.
            %\paragraph{}

            %Son pôle de recherche.
            %\paragraph{}
            %Ses compétences, son marché.

\section{Sujet du stage}
Représenter le trafic aérien de la \textsc{Fir}\footnote{FIR: région d'information de vol, de l'anglais Flight Information Region) est une division de l'espace aérien permettant de faciliter le travail des organismes du contrôle aérien.} de Tahiti dans le logiciel Google Earth. Le logiciel GoogleEarth permet de représenter un espace tridimensionnel et de placer, à l’intérieur du logiciel, des indicateurs tels que des marqueurs de position, des lignes et des polygones. Le logiciel GoogleEarth utilise des fichiers externes de type \textsc{Xml}\footnote{\textsc{Xml}: Extensible Markup Language («langage extensible de balisage»), est un langage informatique de balisage générique.} pour tracer ou représenter ces graphismes. Le format de ces fichiers est ouvert et publié par Google. 

Il s’agit, à partir des traces fournies par le système de contrôle de Tahiti, d’afficher le trafic aérien circulant dans la \textsc{Fir} à des fins d’analyse, de vérification de trajectoire, de mesure de distance…

    
\section{Présentation de l’environnement}
    \subsection{\textsc{Dsna/Dti}}
La Direction des Services de la Navigation Aérienne est chargée de rendre le service de navigation aérienne pour l’État français. A ce titre, la \textsc{Dsna} est responsable de rendre les services de circulation aérienne, d’information aéronautique et d’alerte sur le territoire national et ceux d’outre-mer (\textsc{Dom, Tom , Pom}). La \textsc{Dsna} s’appuie sur deux directions pour exécuter cette mission:
\begin{itemize}
    \item La Direction des opérations ou \textsc{Do},
    \item la Direction de la Technique et de l’Innovation ou \textsc{Dti}.
\end{itemize}\medskip
La \textsc{Do} est l’acteur opérationnel du contrôle aérien tandis que la \textsc{Dti} est chargée du volet technique. Celui-ci consiste à réaliser ou acquérir les systèmes qui participent à l’exercice du contrôle aérien. Il s’agit de systèmes informatiques permettant d’assister le contrôleur dans ses activités, de chaînes radios pour communiquer avec les aéronefs, de systèmes de traitement de l’information météorologique…

La \textsc{Dti} réalise également de nombreuses études pour traiter les besoins des utilisateurs et les évolutions réglementaires. La \textsc{Dti} réalise le déploiement et le support opérationnel des systèmes qu’elle acquiert ou réalise. 

Enfin la \textsc{Dti} fait viser ses systèmes, procédures et formation par l’autorité de surveillance nationale (Direction de la Sécurité de l'Aviation Civile ou \textsc{Dsac}).

La DTI est structurée en domaines qui sont chacun en charge de plusieurs pôles de compétences :
\begin{itemize}
    \item Recherche et développement, R \& D
    \item Exigences opérationnelles des systèmes, \textsc{Eos}
    \item Gestion du trafic aérien, \textsc{Atm}
    \item Communication, navigation, surveillance, \textsc{Cns}
    \item Déploiement et Support Opérationnel, \textsc{Dso}
\end{itemize}\medskip

Chaque pôle couvre un ensemble de fonctions et d’expertises.

Pôle \textsc{Atm/Vig} :
\begin{itemize}
    \item Le pôle « Vol et information générale » (\textsc{Vig}) est responsable de la maîtrise d’ouvrages systèmes de traitements des plans de vol et informations générales, à ce titre, le pôle assure le suivi industriel de leur réalisation ou de leur acquisition. Le pôle \textsc{Vig} est également chargé de leurs maintiens en condition opérationnelles lorsqu’ils sont déployés.
    \item Le pôle \textsc{Atm/Vig} est notamment responsable de la maîtrise d’ouvrages de systèmes déployés en outre-mer. L’aéroport de Tahiti (Polynésie française) a récemment été modernisé avec un système entièrement acquis auprès d’un industriel, couplé à un radar dans le cadre du projet \textsc{Tiare}, qui s’est terminé en 2009.
\end{itemize}\medskip


\section{Le site de Tahiti}
    \subsection{Objectif contrôle aérien}
Le contrôle aérien est un ensemble de services \nref{servicesaerien} rendus par les contrôleurs aériens aux aéronefs afin d'aider à l'exécution sûre, rapide et efficace des vols. Les services rendus sont au nombre de trois, appelés « services de la navigation aérienne », dans les buts de:
\begin{itemize}
    \item prévenir les collisions entre les aéronefs et le sol ou les véhicules d'une part, et les collisions en vol entre aéronefs d'autre part (autrefois appelés « abordages »). Il consiste aussi à accélérer et ordonner la circulation aérienne,
    \item de fournir les avis et renseignements utiles à l'exécution sûre et efficace du vol : informations météorologiques, information sur l'état des moyens au sol de navigation, information sur le trafic (quand le service de contrôle n'est pas assuré dans cette zone),
    \item de fournir un service d'alerte pour prévenir les organismes appropriés lorsque les aéronefs ont besoin de l'aide des organismes de secours et de sauvetage, et de prêter à ces organismes le concours nécessaire.
\end{itemize}\medskip

    \subsection{Les services de la circulation aérienne\label{servicesaerien}}
Comme nous l'avons vu plus haut, le contrôle aérien rend plusieurs services. Nous allons voir ces services plus en détails.

        \subsubsection{Le service de contrôle}
Le service de contrôle est assuré dans les buts suivants:
\begin{itemize}
    \item Prévenir les collisions entre aéronefs ou entre un aéronef et un obstacle
    \item Accélérer et ordonner la circulation aérienne
\end{itemize}\medskip

Le plus important reste donc la sécurité des vols. Le contrôleur s'assure que rien n'arrivera à l'aéronef pendant son vol par des causes extérieures (autre avion, obstacle), et qu'il arrivera à sa destination le plus vite possible. En outre le contrôleur est responsable de la sécurité des vols sous sa juridiction.

Les moyens qu'utilise le contrôleur pour prévenir les abordages sont la séparation (anciennement l'espacement) et l'information de trafic.
\begin{itemize}
    \item La séparation consiste à ménager entre deux aéronefs une distance minimale, garantissant la sécurité de ces deux avions.
    \item L'information de trafic est une information précise sur la position d'un autre aéronef pouvant se rapprocher dangereusement. Le pilote peut ne pas voir qu'un avion se rapproche, l'information de trafic l'aide à voir, afin de permettre au pilote d'éviter l'aéronef conflictuel.
\end{itemize}\medskip

        \subsubsection{Le service d'information}
Le service d'information de vol est assuré sur tout le territoire français. En espace aérien contrôlé, il est assuré par le service de contrôle. Dans les espaces aériens non contrôlés, il est assuré par un organisme UIV (dans les CRNA) et SIV (dans les approches) en vol, ou AFIS sur un aérodrome.
Il consiste à délivrer aux aéronefs les renseignements et avis nécessaires à l'exécution sûre et efficace du vol. Ces renseignements peuvent être (liste non exhaustive) :
\begin{itemize}
\item Météorologiques : conditions météo sur un terrain, présences d'orages…
\item Information sur le trafic (à ne pas confondre avec l'information de trafic) : information sur un trafic connu ou inconnu, en fonction des éléments disponibles, pouvant interférer avec un aéronef.
\item État des aides à la navigation
\item État des équipements sol d'un terrain
\item Amendements de plan de vol
\item Information sur la position, aide aux pilotes perdus
\item Autres ...
\end{itemize}\medskip

        \subsubsection{Le service d'alerte}
Le service d'alerte est aussi vaste que naturel. Il consiste à répondre à tous les besoins des avions qui se disent en détresse, ou dont on peut penser qu'ils sont en détresse. Ce service recouvre des domaines très variés :
\begin{itemize}
\item Si un avion a déposé un plan de vol, et que le contrôle à l'arrivée a reçu confirmation qu'il a bien décollé, il doit surveiller que l'avion arrive bien à destination aux alentours de l'heure prévue, et lancer des recherches si ce n'est pas le cas.
\item Si un avion ne répond plus à la radio et disparaît du radar, le contrôleur doit vérifier si l'aéronef a eu un problème et s'il s'est écrasé ou posé en urgence. Il déclenche alors les secours pour rechercher l'épave et secourir les occupants.
\item Si un aéronef s'écrase sur la piste ou à proximité de l'aérodrome, il déclenche immédiatement les secours et coordonne leur action jusqu'à l'arrivée des renforts.
\item Si un pilote signale avoir des problèmes avec son aéronef de nature à entraver le bon déroulement du vol, le contrôleur peut lui donner une priorité absolue à l'atterrissage en écartant tous les autres aéronefs.
\item Si le contrôleur sait ou soupçonne qu'un aéronef est détourné, il prévient les autorités compétentes et leur apporte tout le secours nécessaire.
\end{itemize}\medskip

D'une manière générale, ce service est une autorisation légale à porter secours par tous les moyens à un pilote en difficulté. Tout être humain le ferait, mais le service d'alerte donne au contrôleur une justification légale pour retarder ou dérouter certains aéronefs afin de porter secours à un autre.

    \subsection{La zone de contrôle de Tahiti: \textsc{Fir}}
        \subsubsection{Le transport aérien}
L’île de Tahiti est desservie par l'Aéroport International Tahiti Faa'a, situé à 5km au Sud-Ouest de Papeete. Inauguré en 1961, et détenu à 57\% par le Territoire de la Polynésie Française10, c’est le plus important aéroport de la Polynésie française, et le seul aéroport international du territoire. Il s’agit donc de l’unique point d’entrée pour l’immense majorité des visiteurs mais également pour les habitants des autres îles de la Polynésie française.

L’aéroport assure les liaisons avec une dizaine de destinations internationales : Los Angeles, Paris, Auckland, Sydney, Tokyo, Rarotonga, Santiago, l’Île de Pâques, Noumea et Honolulu10. Conscient de l’importance des liaisons aériennes internationales dans le développement économique de l’île et du pays, le gouvernement a inauguré en 1998 sa propre compagnie aérienne : Air Tahiti Nui (ATN), qui dessert aujourd’hui 5 destinations à partir de Tahiti : Paris, Los Angeles, Tokyo, Auckland, Sydney.

Concernant le réseau domestique, l’aéroport dessert l’ensemble des archipels de la Polynésie. Air Tahiti est la seule compagnie à desservir régulièrement les îles polynésiennes, assurant la liaison avec une quarantaine d’îles et d’atolls. L’île de Moorea, située à 7 minutes de vol de Tahiti est desservie par Air Moorea, une filiale de la compagnie domestique d’Air Tahiti. L’aéroport de Tahiti est la plaque tournante du trafic aérien, puisque la majorité des destinations sont uniquement desservies par l’aéroport de Tahiti. La centralisation du réseau aérien accentue donc l’attraction et l’influence de Tahiti et de l’agglomération de Papeete sur le reste des îles polynésiennes.

        \subsubsection{Etendue de la \textsc{FIR}\label{Fir}}
\begin{figure}[!h]
\center
\includegraphics[width=15cm]{images/fir.png}
\caption{\textsc{Fir} de Tahiti dans la région Pacifique-Sud.}
\label{stats}
\end{figure}
La région d'information de vol de Tahiti ou « Flight Information Region » (\textsc{Fir} Tahiti) s'étend bien au-delà des eaux territoriales et déborde même sur l'hémisphère nord pour atteindre le parallèle 03\degre30' Nord, soit près de 3700 km de nord au sud et à peu près autant d'est en ouest, couvrant environ 12,5 millions de km$^2$.

Cette \textsc{Fir} constitue le volume au sein duquel la fourniture des services de la circulation aérienne sont assurés sous la responsabilité de l'administration française. Ces services comprennent :
\begin{itemize}
    \item alerte et sauvetage
    \item information de vol
    \item contrôle de la circulation aérienne
\end{itemize}\medskip
La FIR Tahiti est fréquentée par différents types de trafic :
\begin{itemize}
    \item les vols transpacifiques (entre la côte ouest des Etats-Unis et la Nouvelle-Zélande ou l'Australie)
    \item la desserte internationale de Tahiti (depuis et vers les Etats-Unis, la Nouvelle-Zélande, l'Australie, le Japon et le Chili)
    \item les vols intérieurs (desserte domestique des 47 aérodromes de Polynésie Française)
\end{itemize}\medskip

Plus de 40 contrôleurs aériens travaillent 24h/24 et 7j/7 dans la tour de contrôle de Tahiti-Faa'a.

Plus de 20 contrôleurs travaillent sur les aérodromes contrôlés des îles.

En 2006, le centre de contrôle a contrôlé 102 132 mouvements (+2,5\%), dont 71477 mouvements \textsc{Ifr}\footnote{\textsc{Ifr}: (soit, en anglais, Instrument flight rules) règles de vols aux instruments} et 30655 mouvements \textsc{Vfr}\footnote{\textsc{Vfr}: Visual flight rules, nom anglais de « Vol à vue »}.

        \subsubsection{La zone \textsc{Aci}:\label{Aci}}
Une fonction de contrôle spécifique, nommée \textsc{Aci}\footnote{\textsc{Aci}: Area Common Interest, soit une zone d'intérêts commun} ou zone \textsc{Aci}, a été développée dans le système \textsc{Eurocat-X} pour répondre à des besoins de contrôle. Il s’agit d’une zone particulière limitrophe à la \textsc{Fir} \nref{Fir} de Tahiti, dont la limite se situe à 50 miles nautiques de la \textsc{Fir}. La zone \textsc{Aci} encercle la \textsc{Fir}. Il est à noter que cette zone n’est pas sous la responsabilité des contrôleurs aériens français, cependant, les vols pénétrant dans cette région sont visualisés par le système Eurocat-X 

Ainsi en visualisant le trafic aérien dans la zone \textsc{Aci}, les contrôleurs peuvent maintenir les séparations entre les aéronefs. C'est-à-dire vérifier que les vols qui sont à l’extérieur et longent la \textsc{Fir} de Tahiti sont séparés des vols évoluant dans cette \textsc{Fir}.

    \subsection{Le système de contrôle: \textsc{Tiare}}
\textsc{Tiare} est le nom donné au projet qui a débuté en 2007 pour s'achever fin 2010. Ce projet visait à moderniser les moyens informatiques de contrôle du centre de Tahiti, de remplacer les systèmes vieillissants de visualisation du trafic (\textsc{Vivo}) et de gestion de plans de vol et d'informations générales (\textsc{Sigma}). La \textsc{Dti} a fait l'acquisition de deux systèmes différents pour couvrir l’ensemble des missions dévolue aux personnels du bureau de piste et du contrôle aérien.

Les situations de contrôle auxquelles doivent face les contrôleurs sont multiples, il y en a en effet à traiter les spécificités du contrôle océanique, du contrôle d’approche et inter-îles. Le système \textsc{Tiare} est construit à partir de plusieurs « produits sur étagère » :
\begin{itemize}
  \item \textsc{Eurocat-X}, système en charge du traitement radar et de la gestion plans de vols.
  \item \textsc{Atalis}, système en charge de la préparation des vols, de la gestion des \textsc{Notam}, et de la présentation d’informations générales au contrôleur tour et approche.
\end{itemize}\medskip

Les systèmes \textsc{Eurocat-X} et \textsc{Atalis} sont connectés au commutateur \textsc{Cagou}, raccordé aux liaisons externes (\textsc{Rsfta}). \textsc{Atalis} reçoit également des informations météorologiques en provenance du système local d’acquisition de ces données appelé \textsc{Caobs}. \textsc{Eurocat-X} est raccordé au radar secondaire du mont Marau et au réseau \textsc{Acars}.

%PREVOIR SHEMA

La zone Aci \nref{Aci}, a été développée spécifiquement dans le système \textsc{Eurocat-X} pour répondre à des besoins de contrôle.


    \subsection{L'\textsc{Ads-C}}

Avec l'\textsc{Ads-C} (Automatic Dependant Surveillance - Contract), l'avion utilise ses systèmes de navigation satellitaires ou inertiels pour automatiquement déterminer et transmettre au centre responsable sa position et d'autres informations.

Les informations transmises via l'\textsc{Ads-C} peuvent être:
\begin{itemize}
\item La position de l'avion,
\item Sa route prévue,
\item Sa vitesse (sol ou air),
\item Des données météorologiques (direction et vitesse du vent, température...).
\item Les informations de l'\textsc{Ads-C} sont transmises via des communications point à point, par \textsc{Vhf} ou par satellite. Les systèmes sol et embarqués négocient les conditions suivant lesquelles ces transmissions s'effectuent (périodiques, sur événement, à la demande, ou sur urgence).
\end{itemize}\medskip

L'\textsc{Ads-C} est typiquement utilisé dans les zones désertiques ou océaniques où il n'y a pas de couverture radar.

Les avantages de l'\textsc{Ads-C} sont :
\begin{itemize}
\item L'utilisation pour la surveillance des zones sans couverture radar.
\item La transmission de l'information "route prévue".
\item La liaison de données air/sol (comme pour le Mode S et l'\textsc{Ads-B}).
\end{itemize}\medskip

L'inconvénient de l'\textsc{Ads-C} est qu'il dépend entièrement de l'avion et de la correction des données qu'il transmet.






\chapter{Expression du besoin et gestion de projet}
%Expression du besoin par les clients
%Énumère le besoin,
%micro analyse,
%Client dirigiste (python , google earth) mais sans besoin précis pour l’utilisation du produit.
%Risque que cela ne marche pas

\section{Expression du besoin\label{besoins}} 
    \subsection{Les besoins initiaux}
L’objectif initial était de pouvoir réaliser un logiciel banalisé et ergonomique permettant de représenter l’ensemble des données de contrôle (repères, balises, secteurs ...) afin de pouvoir visualiser le trafic aérien circulant dans la \textsc{Fir} et la zone \textsc{Aci}.
Les bénéfices attendus de cet outil sont :
\begin{itemize}
\item l’amélioration de l’analyse et de la compréhension visuelle du trafic aérien de Tahiti,
\item la possibilité d’élaborer de statistiques à partir des fonctions de calcule du logiciel,
\item une aide dans le travail de définition des points d’entrée dans la zone \textsc{Aci} que réalise le service de contrôle de Tahiti.
\end{itemize}\medskip

    \subsection{L’évolution des besoins}
Au début du projet des besoins ont été définis. Nous verrons par la suite comment ceux-ci ont pu évoluer. Il faut noter que le client est assez dirigiste, il a déjà vu ce produit pour d'autres applications et a donc une vue globale de ce qu'il souhaite en sortie. A savoir:
\begin{itemize}
    \item Une application étant basée sur le logiciel \textsc{Google Earth}.
    \item Python comme langage de programmation.
\end{itemize}\medskip

L’objectif du choix de ces outils était aussi pour le client l’assurance de proposer à l’issue du stage une maquette complètement fonctionnelle. C'est-à-dire qu’il fallait déjouer la difficulté technique, comme la représentation du trafic sur une sphère, pour se concentrer sur les besoins suscités par cet outil. En outre la durée du stage et la part consacrée à la rédaction du rapport de stage ne permettaient pas d’innover en créant un logiciel de toute pièce.

C'est pourquoi nous avons orienté notre gestion de projet vers une méthode dite agile \nref{extreme}. Cette méthode nous permettra de redéfinir les besoins tout au long du projet en fonction de ce qui a déjà été réalisé. Et ainsi obtenir un produit correspondant au mieux à ce que le client aurait pu espérer.

Lors du lancement du projet les besoins étaient:
\begin{itemize}
    \item Représenter le trafic aérien déposé par les plans de vol dans la zone de contrôle de \textsc{Tahiti} dans \textsc{Google Earth}.
    \item Visualiser la configuration de la plate-forme \textsc{Tiare} (zones de contrôles, points remarquables ...)
\end{itemize}\medskip

Au fur et à mesure de la progression et des possibilités du logiciel, le client a affiné ses besoins et a rajouté les éléments suivants:
\begin{itemize}
    \item Représenter le trafic aérien en fonction du temps
    \item Définir approximativement l'heure d'entrée de et sortie des avions dans la \textsc{Fir} \nref{Fir} en fonction de leurs plans de vol déposés.
    \item Visualiser le vol des avions en temps réel grâce aux données \textsc{Ads-c}\footnote{Avec l'\textsc{Ads-c} (Automatic Dependant Surveillance - Contract), l'avion utilise ses systèmes de navigation satellitaires ou inertiels pour automatiquement déterminer et transmettre au centre responsable sa position et d'autres informations.}.
    \item Visualiser le positionnement des avions estimé par le système \textsc{Tiare} entre deux reports \textsc{Ads} afin de visualiser l'interprétation des données reçues par le système.
    \item Différencier les types de vols en quatre catégories: Entrant, Sortant, Transit, Interne. 
\end{itemize}\medskip

    \subsection{Les risques}
Lorsque l'on a comme projet de réaliser une application qui a déjà été réalisée par le passé, nous avons une base sur laquelle se référencer (en terme de méthode, de temps, de coûts). Hors sur un projet tel que le nôtre ou même aucun prototype n'a encore été réalisé le risque que cela ne fonctionne pas est très élevé.

C'est pour cela qu'une méthode de gestion de projet, dite agile et décrite ci-dessous \nref{extreme}, a été utilisée. Cette méthode nous a permis d'avancer petit à petit afin de susciter des besoins "réalisable". 

Il faut aussi prendre en compte les défauts de la méthode agile qui suscite un besoin sans fin, une nécessité d'avoir un groupe restreint de personne et nécessité d’avoir un expert pour guider.










%besoin en entrée à méthode « agile » - \\
%description de la méthode – \\
%application et différence par rapport // DTI +\\
%enrichir le produit si cela marche\\
%Avantage/inconvénient Extreme programming :\\
%Revue logicielle (validations qui permettront de faire évoluer le produit)
\section{Gestion de projet}
    \subsection{Choix de la méthode de gestion de projet}
Comme nous l'avons vu précédemment, les besoins ne sont pas clairement définis dès le début. Il était donc difficile de pouvoir établir des spécifications explicites dans le but de mettre en place un cycle en V \nref{cyclev}. Nous avons donc choisis une méthodologie de gestion de projet différente de celle appliquée en temps normal à la \textsc{Dti}.

Cette méthodologie devait nous permettre de débuter un projet sans en connaître réellement l'aboutissement final tout en gardant de la rigueur et de l'organisation. Nous avons donc décidé d'utiliser une méthode agile\footnote{Les méthodes Agiles sont des groupes de pratiques pouvant s'appliquer à divers types de projets, mais se limitant plutôt actuellement aux projets de développement en informatique (conception de logiciel). Les méthodes Agiles se veulent plus pragmatiques que les méthodes traditionnelles. Elles impliquent au maximum le demandeur (client) et permettent une grande réactivité à ses demandes. Elles visent la satisfaction réelle du besoin du client et non les termes d'un contrat de développement. }. Après quelques recherches et comparaisons nous nous sommes orientés sur l'extreme programming \nref{extreme}. Nous allons donc vous décrire cette méthodologie et la comparer avec le système utilisé habituellement.

    \subsection{L' Extreme Programming\label{extreme}}

\begin{figure}[!h]
\center
\includegraphics[width=15cm]{images/xp.png}
\caption{Cycle de l'Exreme Programing.}
\label{XP}
\end{figure}

L'Extreme Programming a été inventée par Kent Beck, Ward Cunningham et Ron Jeffries pendant leur travail sur un projet « C3 » de calcul des rémunérations chez Chrysler. Kent Beck, chef de projet en mars 1996 commença à affiner la méthodologie de développement utilisée sur le projet. La méthode est née officiellement en octobre 1999 avec le livre Extreme Programming Explained de Kent Beck. "Wikipedia".

Dans les méthodes traditionnelles, les besoins sont définis et souvent fixés au départ du projet, ce qui accroît les coûts ultérieurs de modifications. Extreme programming s'attache à rendre le projet plus flexible et ouvert au changement en introduisant des valeurs de base, des principes et des pratiques.

L'Extreme Programming repose sur des cycles rapides de développement (des itérations de quelques semaines voir dans notre cas quelques jours seulement) dont les étapes sont les suivantes:
\begin{itemize}
\item une phase d'exploration détermine les scénarios clients qui seront fournis pendant cette itération,
\item la transformation des scénarios en tâches à réaliser et en tests fonctionnels,
\item lorsque tous les tests fonctionnels passent, le produit est livré.
\end{itemize}\medskip

Lorsqu'une tâche est terminée, les modifications sont immédiatement intégrées dans le produit complet. On évite ainsi la surcharge de travail liée à l'intégration de tous les éléments avant la livraison. Les tests facilitent grandement cette intégration: quand tous les tests passent, l'intégration est terminée.

Le cycle se répète tant que le client peut fournir des scénarios à livrer (cf. Fig. \vref{XP}). Généralement le cycle de la première livraison se caractérise par sa durée et le volume important de fonctionnalités embarquées. Après la première mise en production, les itérations peuvent devenir plus courtes (par exemple la séparation des plans de vol en catégories tel que: transit, interne ...)

Pour résumer, cette méthode nous amène à réaliser la boucle suivante:
\begin{itemize}
\item Analyse du besoin
\item Expression des spécifications
\item Réalisation technique
\item Test de la réalisation
\item Revue logicielle (validations qui permettront de faire évoluer le produit)
\end{itemize}\medskip

    \subsection{Le cycle en V}

\begin{figure}[!h]
\center
\includegraphics[width=12cm]{images/cyclev.png}
\caption{Les phases à travers le temps et le niveau de détails.}
\label{cyclev}
\end{figure}

Le modèle du cycle en V est un modèle conceptuel de gestion de projet étudié pour résoudre le problème de réactivité du modèle en cascade. Il permet, en cas d'anomalie, de limiter un retour aux étapes précédentes. Les phases \figref{cyclev} de la partie montante doivent renvoyer de l'information sur les phases en vis-à-vis lorsque des défauts sont détectés, afin d'améliorer le logiciel.

Le cycle en V est devenu un standard de l'Industrie logicielle depuis les années 1980 et depuis l'apparition de l'Ingénierie des Systèmes est devenu un standard conceptuel dans tous les domaines de l'Industrie. Le monde du logiciel ayant de fait pris un peu d'avance en termes de maturité, on trouvera dans la bibliographie courante souvent des références au monde du logiciel qui pourront s'appliquer au système.

Les étapes qui constituent cette méthode sont les suivantes :
\begin{itemize}
    \item Analyse des besoins et faisabilité
    \item Spécification logicielle
    \item Conception architecturale
    \item Conception détaillée
    \item Codage
    \item Test unitaire
    \item Test d'intégration
    \item Test de validation (Recette Usine, Validation Usine - VAU)
    \item Recette (Vérification d'Aptitude au Bon Fonctionnement - VABF)
\end{itemize}\medskip
 
Dans notre cas, avec ces besoins indéfinis, cette méthodologie engendrerait un risque. Ce risque serait que le logiciel final ne fonctionne pas ou ne reponde pas aux attentes du client.

\subsection{Les avantages et inconvénients}
Pour notre cas, les avantages de cette méthode sont les suivants:
\begin{itemize}
    \item Enrichir le produit à chaque itération du cycle. Ci le logiciel est fonctionel, le client peut visualiser immédiatement les besoins qui étaient superficiels (dont il n'avait pas réellement besoin) et au contraire découvrir de nouveaux besoins.
    \item Rediriger rapidement la conduite du projet. Si le client souhaite rediriger son projet, ceci peut être fait dans les meilleurs délais (changement d'objectifs ou de priorités)
%    \item ... a completer.
\end{itemize}\medskip
 
Cette méthode implique tout de même un certain nombre d'inconvénients tels que:
\begin{itemize}
    \item Le client doit être disponible afin de faire avancer le projet. Chaque validation est vue avec le client et c'est celui-ci qui donne les nouveaux besoins. Ce qui implique que ci celui-ci n'est pas disponible, le projet peut vite être bloqué. 
    \item Le projet peut vite dériver. Ce type de méthode requiert des personnes compétentes, aussi bien au niveau Maitre d'ouvrage, qu'au niveau maitre d'œuvre. Il est facile de s'égarer c'est pourquoi une organisation et une rigueur doivent être entretenues tout au long du projet.
    \item Lors du début du projet, on manque cruellement de spécifications. On se lance alors dans le développement sans analyse complète.    
%    \item ... a completer.
\end{itemize}\medskip







    


\chapter{Réalisation technique}
%Réalisation technique
    %Contexte technique opérationnel
        %Plan de vol et reports ADSC
        %Plate-forme TIARE (production de log fichiers texte)
    %Base de travail
        %Python : librairies, IDE, Linux
        %Google Earth
    %Conception Produit
            %Le programme réalisé et ses fonctions
                %Analyse DATASET
                %Analyse du trafic
    %Problèmes techniques rencontrés
            %Rotondité, intersection, zip, optimisation dans google earth….



%Avant de passer à la pratique un apprentissage théorique à du être réalisé.
\section{Architecture du logiciel}
    \subsection{L'interface \textsc{Google Erath}}
\textsc{Google Earth} dispose d’une interface graphique qui sera mise à profit pour:
\begin{itemize}
\item représenter les points remarquables (points nommés, points de coordination, etc.),
\item représenter les espaces de contrôles,
\item représenter le trafic aérien.
\end{itemize}\medskip

Ces données sont, soit statiques, soit dynamiques.
\begin{description}
\item[Statiques:] affichage de points fixes et affichage des espaces ou trajectoire plan de vol. 
\item[Dynamique:] représentation du trafic aérien en fonction de coordonnées mises à jour et en fonction du temps.
\end{description}\medskip

    \subsection{Gestion de l'affichage}
Google Earth peut être enrichi de données externes via un fichier descriptif de données (\textsc{Kmz}).

Ce fichier Kmz n'est autre qu'un fichier Zip compressant un fichier "doc.kml" ainsi que les fichiers vers lesquels ils pointent. Ce fichier "doc.kml" a pour but de regrouper les fichiers à utiliser (comme le ferait une liste de lecture pour les fichiers Mp3). Il indique donc à \textsc{Google Earth} de charger les fichiers suivants:
\begin{description}
\item[CharacteristicsPoints.kml:] Le fichier contenant les points remarquables
\item[Fir.kml:] pour les zones de contrôle et la zone \textsc{Aci} 
\item[Routes.kml:] regroupe toutes les routes définies
\item[Fpl.kml:] affiche les plans de vol déposés
\item[Ads.kml:] affiche le trafic aérien réel reçu par l'\textsc{Ads-c}
\end{description}\medskip

Les fichiers KML sont fabriqués à partir des fichiers de configuration et de traces définis dans le système \textsc{Eurocat-X}. 

Le système EurocatX est configuré au moyen de fichiers de configuration statique. Ces fichiers seront «parsés» pour fabriquer les fichiers: "CharacteristicsPoints.kml", "Fir.kml" et "Ads.kml".

Ce système est également constitué de fichiers de traces qui enregistre les informations des vols. Ces fichiers seront «parsés» pour fabriquer les fichiers: "Ads.kml" et "Fpl.kml".

    \subsection{Les modules}
Au final le logiciel se compose des modules suivants:
\begin{description}
\item[Manu:] Fichier principal, peut être considéré comme l'exécuteur. \nref{pyManu}
\item[modules/Ads:] Met en mémoire les informations sur les reports \textsc{Ads}. \nref{pyAds}
\item[modules/Aoi:] Permet de définir tout les volumes utilisés pour concevoir les zones de contrôle. \nref{pyAoi}
\item[modules/CharacteristicPoints:] Met en mémoire tous les points remarquables disponibles sur le système. \nref{pyCP}
\item[modules/Convertion:] Regroupe plusieurs fonctions utilisées pour convertir des données (ex: utilisé pour convertir les coordonnées). \nref{pyConvertion}
\item[modules/Fdp:] définit et met en mémoire toutes les zones de contrôle. \nref{pyFdp}
\item[modules/Fpl:] définit et mets en mémoire les plans de vol. \nref{pyFpl}
\item[modues/GetOfFiles:] Coordonne la récupération des données, c'est lui qui va chercher la configuration et lance les modules tels que: Aoi, Fdp ou encore Fpl. \nref{pyGOF}
\item[modules/KML:] Ce module est utilisé pour mettre en forme les fichiers \textsc{Kml} à l'aide des données reçues en entrée (par exemple pour un point il reçoit sa description, ces coordonnées, son nom ...). \nref{pyKML}
\item[modules/MakeKML:] C'est le module qui exploite toutes les données en mémoire et crée les fichiers \textsc{Kml}. \nref{pyMakeKML}
\item[modules/MakeKMZ:] Récupère les fichiers \textsc{Kml} pour les regrouper en un fichier \textsc{Kmz} plus maniable. \nref{pyMakeKMZ}
\item[modules/Routes:] définit et met en mémoire les routes. \nref{pyRoutes}
\item[modules/usualFonction:] Regroupe plusieurs fonctions régulièrement utilisées, cela évite de les réécrire dans chaque module les utilisant. \nref{pyusualFonction}
\end{description}
Chaque module est décrit avec plus de précisions en annexe.

Le fichier principal (Manu.py) se lance à partir de la ligne de commande: "Python~Manu.py" dans un système ou Python est installé et correctement configuré.

Les fichiers ".asf" contenant la configuration de l’eurocatx sont placés dans le répertoire "SurcesAsf" alors que les fichiers contenant les traces sont dans le répertoire "Sources". 

Tout les modules sont appelés automatiquement, se qui signifie qu'après avoir renseigné le fichier de configuration et exécuté le fichier principal, aucune action n'est nécessaire pour concevoir le fichier \textsc{Kmz}. Il n'y aura donc plus qu'à exécuter ce fichier \textsc{Kmz} en l'ouvrant depuis \textsc{Google Earth}. 

L'exécution du programme génère aussi des fichiers de log. Ces fichiers seront utiles pour visualiser les message corrompu (mal interprété) ou encore des points non définis.

\section{Le contexte technique opérationnel}

    \subsection{\textsc{EurocatX}}
Il faut bien comprendre comment marche le système afin de bien visualiser d'où proviennent les informations. Comme décrit grossièrement dans le schéma (cf. Fig. \vref{eurocatx}),
\begin{figure}[!h]
    \center
    \includegraphics[width=15cm]{images/SchemaControle.png}
    \caption{Schématisation du système \textsc{EurocatX} au niveau des tours de contrôle}
    \label{eurocatx}
\end{figure}
\textsc{EurocatX} récupère les informations sur les plans de vol par l'intermédiaire de \textsc{Cagou}\footnote{\textsc{Cagou}: nom donné au commutateur \textsc{Rsfta}}. Il récupère aussi le positionnement émis par l'avion à l'aide de la transmission Satellite, \textsc{Vhf}\footnote{\textsc{Vhf}: Very High Frequency, soit une bande radio de très haute fréquence} ou des données radars lors de son approche. Le système \textsc{EurocatX} donne un accès à la bureautique protégé par un par-feu (FireWall) afin de rendre disponible sur ce réseau un certain nombre d'informations. Dans notre cas nous y récupérerons:
\begin{itemize}
    \item toutes les données de configurations du système tels que les noms et coordonnées des balises référencées, la position des zones de contrôles et des zones \textsc{Aci} ou encore les routes utilisées pour décrire les plans de vols.
    \item Les fichiers de log du Commutateur \textsc{Cagou} afin de pouvoir exploiter les plans de vol reçus par le réseau \textsc{Rsfta}.
    \item Tous les reports \textsc{Ads} reçus par satellite et traités par le système.
\end{itemize}\medskip 
Le système envoie les informations récoltées et celles calculées aux visues\footnote{Visue: Nom pour décrire les ordinateur utilisés pour visualiser les données de contrôles} situées dans la tour de contrôle au niveau de la Vigie ou de la salle \textsc{Ccr} ainsi que de la position déportée à \textsc{Moréa}.

Les données seront donc récupérées dans les fichiers ".asf" pour la configuration système, dans les fichier du \textsc{Fdp} pour les plans de vol et dans les fichiers du serveur \textsc{Ads} pour les reports ainsi que pour la position calculée des aéronefs.

    \subsection{Le domaine de l'aviation}
Il m'a été nécessaire de prendre connaissance de touts les termes, unité, convention, utilisés dans le domaine aéronautique.

        \subsubsection{Les coordonnées et unités:}
Tout d'abord est vite venu le problème de conversion de coordonnées, j'ai donc du revoir les conversions de coordonnées sphériques ainsi que les conversions de distances.
J'ai également dû, comme expliqué ci-dessous (cf. \vref{mathcoord}), me remémorer les solutions de calculs du point d'intersection de deux arcs de cercles en coordonnées sphériques.

        \subsubsection{Convention:}
Plusieurs conventions on dû être acquises comme celles utilisées par le système \textsc{Tiare} pour décrire les reports \textsc{Ads} ou entre celles utilisées par les compagnies pour le dépôt de plans de vol \bibref{doc4444}.





\section{Base de travail}
    \subsection{Le langage Python}
        \subsubsection{Bien coder:\label{pygood}}
Afin de pouvoir apprendre les bonnes pratiques de la programmation Python j'ai lu un livre intitulé "Programmation Python, conception et Optimisation"\cite{pybook}. Celui-ci m'a permis de pouvoir d'une part revoir ce qui avait été appliqué lors de mes études et d'autre part avoir une vue global sur le langage et ainsi pouvoir prendre du recule lors du codage.

Celui ci m'a par exemple appris le nouveau style de programmation qui part du principe que chaque nouvel objet défini est basé sur un Objet existant, et que par la même occasion tout en python était Objet (même une simple variable booléenne). Ou encore la manière de vérifier si un objet était faux, égale à 0 ou encore une chaîne vide simplement en demandant si il existait (ex:~\texttt{"if~x~!=~0:"}~devient \texttt{"if~not~x:"})

        \subsubsection{Utiliser les expressions régulières:} 
L'apprentissage de l'utilisation des expressions régulières\footnote{Une expression régulière est en informatique une chaîne de caractères que l’on appelle parfois un motif et qui décrit un ensemble de chaînes de caractères possibles selon une syntaxe précise.}, m'a été grandement facilité grâce au site: \url{http://www.dsimb.inserm.fr/}\cite{re} et a la documentation en ligne de Python \bibref{pydoc}. Il s'est avéré après apprentissage que ces expressions régulières auront grandement facilité la faisabilité du projet.

        \subsubsection{L'optimisation:}
Je pourrai citer un passage du livre \bibref{pybook} qui dit:
\begin{quotation}
    Fourni dès le départ avec des modules de tests, Python est un langage agile. Le terme agile est originellement issu de la méthodologie de programmation agile (Beck et Al.), très proche de la programmation itérative. Cette méthodologie, qui réduit les risques liés à la conception de logiciels, introduit entre autres des principes de tests continus du code.
    \raggedleft Vincent \textsc{Lozano}.
\end{quotation}

En effet il m'a été rapidement nécessaire de réaliser des tests, aussi bien pour vérifier que mon code était valide que pour vérifier que celui-ci s’exécutait normalement. Il s'est avéré à plusieurs reprises que certaines parties de mon code étaient très gourmandes en processus. L’apprentissage de fonctions de test de code, tel que le module hotshot décrit plus tard (cf. \vref{perf}), m'a été rapidement nécessaire.

    \subsection{\textsc{Google Earth}}
\textsc{Google Earth} est un logiciel, propriété de la société \textsc{Google}, permettant une visualisation de la terre en 3 dimensions avec un assemblage de photographies aériennes ou satellites. Ce logiciel donne la possibilité de configurer un environnement, ajouter des lignes, des points ou encore des polygones en 3D en passant par des fichiers de configuration au format \textsc{Kml}\footnote{\label{Kml}\textsc{Kml}: Keyhole Markup Language, est un format de fichiers et de grammaires \textsc{Xml} pour la modélisation et le stockage de caractéristiques géographiques comme les points, les lignes, les images, les polygones et les modèles pour l'affichage dans \textsc{Google Earth}, dans \textsc{Google Maps} et dans d'autres applications.}.

Ce format, qui repose sur le \textsc{Xml}\footnote{\textsc{Xml}: Extensible Markup Language («langage extensible de balisage»), est un langage informatique de balisage générique.}, a l'avantage d’être simple à manipuler. Ça sémantique est définie sur le site de google \bibref{gecode} 



\section{Le programme réalisé et ses fonctions}
    \subsection{Le fonctionnement\label{fonctionnement}}
            \paragraph{La configuration:}
Le programme réalisé ne possède pas encore d'interface (IHM) graphique. Il est donc nécessaire de configurer les options à l'aide d'un fichier de configuration (cf. annexe \vref{pyConfig}). Nous pourrons régler par l'intermédiaire de celui-ci:
\begin{itemize}
    \item Les fichiers Kml à recréer ou non, ce qui est utile afin de ne pas avoir à recréer des fichiers statiques (tel que la position des points caractéristiques ou encore des zones de contrôle) à chaque utilisation tout en laissant à l'utilisateur la possibilité de les mettre à jour simplement.
    \item Les différents styles et couleurs.
    \item L'emplacement des fichiers de configuration.
    \item Les descriptions et noms appliqués à chaque catégorie.
\end{itemize}\medskip
            \paragraph{L'exécution:}
Le fichier de configuration renseigné, le programme peut être lancé. Il est possible de le lancer par l'intermédiaire d'un Shell\footnote{Shell: Interface en lignes de commandes}, par l'intermédiaire de l'interface Python ou encore en direct si les informations pour gérer et lancer les fichiers Python ont été renseignées dans le système d'exploitation.
            \paragraph{Le résultat:}
L'exécution du programme réalise une suite d'actions:
\begin{enumerate}
    \item Lire le fichier de configuration afin de déterminer les actions à effectuer.
    \item Lire les fichiers de configuration du système \textsc{Tiare} affin de récupérer toutes les variables nécessaires sous forme d'objets\footnote{Objet: structure de données valuées et cachées qui répond à un ensemble de messages. Cette structure de données définit son état tandis que l'ensemble des messages qu'il comprend décrit son comportement} (ex: points caractéristique ...)
    \item Lire les fichiers de log afin de créer des objets tels que les plans de vol ou encore les reports \textsc{Ads}. Ces objets sont créés non seulement à partir de ces fichiers de log mais aussi à partir des objets créés précédemment (ex: les points des plans de vol désignés par un nom sont convertis en coordonnées à l'aide des points caractéristiques).
    \item Créer les fichiers \textsc{Kml} désignés dans le fichier de configuration à l'aide des objets instanciés.
    \item Créer un fichier \textsc{Kmz} à l'aide de tous les fichiers \textsc{Kml} afin d'avoir un fichier compact et facile a transporter.
\end{enumerate}

    \subsection{L'exploitation dans \textsc{Google Earth}}
L'exécution du programme retourne en résultat un fichier \textsc{Kmz}. C'est ce fichier qui est utilisé pour exploiter les données dans \textsc{Google Earth}. Pour cela il suffit d'ouvrir le fichier à l'aide de ce logiciel.

Nous allons vous présenter quelques exemples d'utilisations de ce logiciel.

        \subsubsection{Vue d'ensemble}
\begin{figure}[!h]
\center
\includegraphics[width=12cm]{images/gevuedensemble.png}
\caption{Vue d’ensemble du trafic dans \textsc{Google Earth}}
\label{gevuedensemble}
\end{figure}
Lors de l'ouverture du fichier la vue est centrée sur la zone de contrôle de Tahiti. Comme vous pouvez le voir \figref{gevuedensemble} nous pouvons distinguer trois zones dans le logiciel:
            \paragraph{La zone de sélection :}
Cette zone, numérotée 1 sur la figure, sert à sélectionner les éléments à afficher ou non. 

Chaque groupe d'éléments est représenté par un dossier. Ainsi il sera plus facile de sélectionner un groupe tel que les routes. Il sera aussi simple de dé-sélectionner un groupe (par exemple groupe Fpl qui contient tous les plans de vol) et de re-sélectionner un seul élément du groupe afin de le visualiser séparément (ex: le plan de vol d'un avion précis).

            \paragraph{L'animation temporelle:}
Sur notre figure cet outil est représenté par le numéro 2. Nous pourrons grâce a lui visualiser l'évolution du trafic dans le temps. Les principales options utiles à notre cas seront:
\begin{itemize}
\item La sélection d'une date et une heure précise afin de visualiser où devrait se trouver un avion, ou encore avoir une vue de tous les vols en cours à cette heure.
\item Un créneau compris entre deux dates et heures. Cette option nous permettra de visualiser un vol sur un partie de son parcours afin d'avoir une vue un peu plus globale. Nous l'avons utilisé lors de l'exemple suivant \figref{gezoom} afin de mieux visualiser les points estimés par le système \textsc{Tiare}.
\end{itemize}

            \paragraph{La vue:}
Cette zone, numérotée 3, nous permet de visualiser notre sélection configurée à l'aide des deux zones citées précédemment.

        \subsubsection{Exploitation des données}
\begin{figure}[!h]
\center
\includegraphics[width=12cm]{images/gezoom.png}
\caption{Zoom sur la déviation de la trajectoire d'un vol par rapport à son plan de vol déposé}
\label{gezoom}
\end{figure}
Nous avons pris en exemple un zoom sur un plan de vol qui a été dévié de sa trajectoire initiale (son plan de vol).
Nous pouvons donc apercevoir sur cette figure \figref{gezoom}:
\begin{description}
\item[Les lignes noires] Ces lignes représentent les plans de vol des avions en cours à la date et l'heure sélectionnées. Ci ceux-ci ne sont pas recouverts par une ligne de couleur plus épaisse, cela signifie que l'avion n'est pas censé se trouver à cet endroit à l'instant défini. 
\item[La ligne verte:] Cette ligne représente le plan de vol déposé. Chaque segment de cette ligne représente ou peut se situer l'avion à l'instant donné.
\item[La ligne blanche:] Cette ligne représente la trajectoire réellement effectuée par l'avion. Cette ligne est définie par les reports \textsc{Ads}.
\item[Les points verts:] Ces points représentent les reports \textsc{Ads} reçus par le système \textsc{Tiare}. Lorsque l'on clic sur l'un de ces points il est possible de voir sa description qui contient le message émis par l'avion.
\item[Les points roses:] Ils représentent les points estimés par le système \textsc{Tiare}. Dans ce cas précis nous constatons que ces points ne correspondent pas avec la trajectoire réellement réalisée. Cette différence peut s'expliquer par le fait que le système utilise les points de reports suivant estimés par l'avion pour définir la position actuelle de l'avion.
\item[Les points oranges:] Ces points représentent les reports suivant estimés (next report) par l'avion cité précédemment.
\item[Les points rouges:] Ces points représentent l'heure d'entrée, estimée par le programme, dans la zone \textsc{Aci} \nref{Aci}. 
\end{description}




\section{Problèmes techniques rencontrés et solutions apportées}
Comme dans tout projet il y a une multitude de problèmes à résoudre. Nous verrons dans cette partie quelques exemples de ces problèmes rencontrés ainsi que la manière dont ils ont été résolus. Cette liste reste bien entendu exhaustive au regard de tous les petits problèmes auxquels nous avons du faire face.

    \subsection{Gestion des erreurs}
            \paragraph{Problématique:}
Le premier problème que nous avons rencontré a été celui de la gestion des erreurs. En effet, de la première mise en route du logiciel jusqu'à la fin du stage des erreurs ont dû être gérées. Deux types d'erreurs sont revenues:
\begin{itemize}
    \item Le premier type d'erreur était par exemple une réaction inattendue du logiciel, On pourrait prendre en exemple la conversion de coordonnées reçues en Système sexagésimal \footnote{(Système sexagésimal : Degrés ( \degres\ ) Minutes ( ' ) Secondes ('' ))} en coordonnées décimales utilisées dans les fichiers KML \vref{Kml}, qui lors des premiers tests donnaient des données erronées.
    \item Le deuxième type était celui dû aux erreurs contenues dans les fichiers de log utilisés pour récupérer les informations. Ces erreurs faisaient effet boule de neige et venait se répercuter dans le fonctionnement du logiciel.
\end{itemize}\medskip

            \paragraph{Résolution:}
La solution au premier problème a été de mettre en place des tests à chaque fonction implémentée ou après avoir réalisé chaque objectif fixé. On appelle cette méthode le test continu du code. Grâce à cela nous allons pouvoir déterminer plus rapidement lors d'une erreur future d'où provient celle-ci. Une méthode simple de la mette en place est de définir un test à réaliser pour valider la fonction ou le code. On détermine donc quelle réaction doit avoir une fonction pour un environnement donné et l'on vérifie si le résultat correspond bien avec celui espéré. (Ex: on a la coordonnée 4530N10045E qui correspond a 45\degres 30' Nord 100\degres 45' Est. On envoie cette variable dans la fonction de conversion et l'on vérifie que le résultat retourné est bien en décimal: 45,5\degres\ en latitude et -100,75 en longitude). Si le résultat est correct la fonction ou le morceau de code est validé, sinon il doit être corrigé.

La solution du deuxième problème a été dans un premier temps d'afficher chaque erreur dans la console, mais cela est vite devenu trop compliqué du fait que la console ne retient par défaut qu'un nombre limité de lignes en mémoire et donc que les lignes trop anciennes sont simplement effacées. On a donc mis en place un système de log permettant, en plus d'avoir accès aux informations les plus anciennes, de pouvoir l'exploiter après avoir fermé la console, effectuer des recherches à l'intérieur et tous avantages que peut apporter un fichier texte. Pour les dernières versions de log, celles-ci sont crées avec des informations relatives aux types d'erreurs et l'emplacement de l'erreur dans le fichier source. Le tout enregistré dans un fichier comprenant la date et l'heure actuel dans le nom afin de pouvoir les différencier de chaque exécution du logiciel. 

    \subsection{Intersection entre plans de vol et zone ACI\label{mathcoord}}
            \paragraph{Problématique:}
Afin de déterminer l'heure d'entrée approximative des avions dans la zone ACI (cf. \vref{Aci}) en fonction de leurs plans de vol déposés, il est nécessaire de déterminer le point d'intersection entre leurs plans de vol et la zone ACI. En théorie cela paraît simple, il suffit de prendre chaque portion du trajet du plan de vol composé de deux points et formant une droite, et  de déterminer si cette droite coupe chaque droite composant la zone ACI. Dans la pratique il s'est avéré que cela était un peu plus compliqué, en effet ces droites sont en réalité des arcs de cercles qui sont composés de deux extrémités définies par des points en coordonnées sphériques (cf. schéma fig. \vref{sphere}).
\begin{figure}
    \center
    \includegraphics[width=5cm]{images/Sphere.png}
    \caption{Représentation grossière de l'intersection de deux arcs de cercle respectivement formés par la trajectoire la plus courte entre deux points situés sur le Globe terrestre}
    \label{sphere}
\end{figure}
            \paragraph{Résolution:}
Étant donné que j'ai effectué un \textsc{Bts} avant d'intégrer l'\textsc{Eigsi} \footnote{\textsc{Eigsi}: École d'Ingénieurs en Génie des Systèmes Industriel située à La Rochelle}, les notions de coordonnées sphériques ne me sont que peu familières. Après avoir en vain cherché sur internet ainsi que dans mon entourage  (maître de stage, collègues de travail) je me suis replié sur un forum de mathématique sur le quel j'ai déposé un sujet explicitant le problème (adresse, cf. bibliographie \cite{forummath}). Un utilisateur nous a donné une solution qui, après connaissance, semble tellement simple qu'on se demande pourquoi personne n'y a pensé. Cette solution consiste a déterminer les plans définis par les deux points aux extrémités de chaque arc et par le centre de la terre (ainsi nous avons forcement la courbe qu'a suivi l'avion sur ce plan). Il faut ensuite déterminer la normale à chacun des plans pour en déduire la droite d'intersection de ces plans (passant par le centre de la sphère). Une fois cette droite acquise il faut définir son vecteur norme et le convertir en coordonnées sphériques. Ce qui nous donne un des points d'intersections de la droite avec la sphère, l'autre étant situé par définition à l'opposé.

Une démonstration valant amplement un long discours, et à titre informatif, voici ce que cela donne en résolution mathématique. Pour cet exemple nous avons deux arcs représentent deux trajectoires définies chacune par deux points A et B (cf. Fig. \vref{sphere}). Chaque point sera défini par une latitude et une longitude.

Nous avons donc:
\begin{itemize}
    \item $lat_{A}$ la latitude de A
    \item $long_{A}$ la longitude de A
    \item $(x_{A}, y_{A}, z_{A})$ les coordonnées cartésiennes de A
    \item $I_{1}$ le point d'intersection \no 1
    \item $I_{2}$ le point d'intersection \no 2
\end{itemize}
Il faut tout d'abord convertir les coordonnées sphériques en vecteurs de coordonnées cartésiennes pour $A$ et $B$:
$$  A=\left\{
\begin{array}{rcl}x_A & = & cos(lat) \times cos(long)\\ y_D & = & cos(lat_{A}) \times sin(long_{A})\\ z_A & = & sin(lat_{A}) 
\end{array}\right.$$
Il faut ensuite déterminer le plan passant par $O$, $A$ et $B$ ayant alors pour équation:
$$ax+by+cz=0$$ où $$\left(\matrix {a\cr b\cr c}\right)= \left(\matrix {x_A\cr y_A\cr z_A}\right)\wedge \left(\matrix {x_B\cr y_B\cr z_B}\right)$$c'est à dire $$\left\{\matrix {a=y_Az_B-z_Ay_B\cr b=z_Ax_B-x_Az_B\cr c=x_Ay_B-y_Ax_B}\right.$$
L'intersection des deux plans de coordonnées $(a,b,c)$ et (a',b',c') contient le point $O$, mais aussi le point $P$ de coordonnées $(x_P,y_P,z_P)$ tel que: $$\left(\matrix {x_P\cr y_P\cr z_P}\right)= \left(\matrix {a\cr b\cr c}\right)\wedge \left(\matrix {a'\cr b'\cr c'}\right)$$
$P$ n'étant pas forcément sur la sphère, il faut trouver un point de la droite $(OP)$ sur cette sphère. Pour cela il suffit de diviser les 3 coordonnées de $P$ par la norme de $\overrightarrow{OP}$:
$$I_{1} = \left\{\matrix {x_P / \sqrt{x_P^2+y_P^2+z_P^2}\cr x_P / \sqrt{y_P^2+y_P^2+z_P^2}\cr x_P / \sqrt{z_P^2+y_P^2+z_P^2}}\right.$$
nous avons donc $I_1$ et son opposé $I_2$, il nous reste donc plus qu'a vérifier si chacun de ces points appartient à un des 2 arcs.

Vous trouverez le code Python correspondant à ces calculs dans la fonction: "verifyIntersection (line, point):" du module "ususalFonction.py" disponible en annexe \vref{pyusualFonction}

    \subsection{Performance du logiciel\label{perf}}
            \paragraph{Problématique:}
Les premiers tests du logiciel ce sont déroulés sur un nombre limité de fichiers (représenté par un nombre limité d'heure de vol), ce afin de pouvoir les valider rapidement. Lors de l'apparition de fichiers plus volumineux (plus de 300Mo de données en entrée, environ 10\% en sortie) c'est posé le problème de performance. Avant optimisation l'ordinateur moulinais des heures avant de pouvoir sortir un fichier. Il a donc fallût optimiser le code afin d'alléger le programme en ressources.

            \paragraph{Résolution:}
En cherchant des conseils dans des forums d'informatique ainsi que dans le livre cité précédemment \bibref{pybook}, nous avons découvert que Python était un langage orienté par les tests et qu'il disposait donc de librairies spécialement conçues pour déterminer les points bloquants d'un programme et les fonctions appelées les plus gourmandes.

La fonction retenue pour repérer ce qui est appelé en anglais les Bottleneck\footnote{Bottlneck: (goulot d'étranglement) point d'un système limitant les performances globales, et pouvant avoir un effet sur les temps de traitement et de réponse.} est la fonction "hotshot" qui a pour but d'analyser un programme dans sa totalité en indiquant notamment les ressources utilisées par chaque fonction appelée. Pour visualiser ce que donne le résultat d'une analyse veuillez vous reporter à la figure \vref{stats}.
\begin{figure}
\center
\includegraphics[width=15cm]{images/stats.png}
\caption{Analyse avec Kcachegrind de l'exécution de la fonction "TestIntersection" du programme dans le but de l'améliorer.}
\label{stats}
\end{figure}

Les bottlenecks repérés, une réécriture des parties bloquantes a dû être effectuée. Cette analyse nous a permis de réduire les ressources et donc le temps d'exécution du logiciel de plus de 80\%.


\chapter{Tests et validation de la réalisation}
Tests et validation de la réalisation
            Démarche pour tester le produit (manque pas des vols)
            Un fichier même vol, mais fichiers avec des vols supplémentaires
            Présentation du rendu
Améliorations continues : à partir des tests, je repars dans le chapitre précédent (réalisation technique + nouveaux besoins (comparaison FPL//ADSC) ou correction)


\chapter{Synthèse}
%- capture de fichiers
%- moulinette pour produire les KML
%- intégration dans googlearth

%Les besoins exprimé

Comme vous l'avez compris, utiliser une méthode agile comme l'extreme programming \nref{xtreme} implique une perpétuelle réecriture des spécifications. Celle ci sont améliorée, réécrite, ajoutée tout au long du projet. C'est pourquoi dans ce rapport serra cité la derniere vertion des spécification.

les spécification du logitielle sont les suivante:
\begin{description}
    \item[Capture de fichiers de configuration] Les points charactéristiques, route, zone de controle et \textsc{Aci}, doivent être récupérés dans les fichiers de configuration du système \textsc{Tiare} afin d'avoir la representation la plus juste de ce que le système a. Il doivent etre gardé en memoire pendent toute l'execution du logiciel afin de pouvoir etre utilisé. Les donnée seront enregistrées dans des objet le temps de l'exection du programme afin de faciliter leur exploitation. La configuration du logiciel doit laisser a l'utilisateur la possibilité de spécifier le chemin du fichier de configuration.
    \item[Capture des données Plan de vol] Les données Plan de vol doivent etre récupérée dans les log du système \textsc{Tiare}. Par contre il doit être possible de les récuperer d'un autre fichier contenent des trame FPL au format normalisé par la norme 4444 (cf Bib. \cite{4444}). Chaque plan de vol serra enregistré dans un objet ayant un identifiant comprenant: L'identifiant de l'avion, son aéroport de depart ainsi que l'heure et le jour de depart. Cet identifiant a pour but de les diferentier et de les réferencer dans le temps. La configuration du logiciel doit laisser a l'utilisateur la possibilité de spécifier le chemin du fichier de log.
    \item[Capture des données \textsc{ADS}] Les données \textsc{ADS} doivent etre récupérée dans les log du système \textsc{Tiare}. Il devra être aussi récupérer dans ces log les points de la position en fonction du temps calculé par le système entre deux report \textsc{Ads}. Les reports \textsc{Ads} et points calculé seront instencié par avion et par vol. L'identifiant de chaque vol sera donc composé de l'identifient de l'avion ainsi que de la date et l'heure du message de login. La configuration du logiciel doit laisser a l'utilisateur la possibilité de spécifier le chemin du fichier de log.
    \item[Les points caractéristiques] Ces points devront être implémenté dans \textsc{Google Earth} avec la possibilité de les afficher ou non. La configuration du logiciel doit laisser a l'utilisateur la possibilité de spécifier la possibilité de réediter ou non le fichier source \textsc{Google Earth}. Ces point seron représenté par un triangle de petite taille.
    \item[Les zonnes de Contrôle et \textsc{Aci}] Les zonnes de controle et zonnes \textsc{Aci} devront être implémenté dans \textsc{Google Earth} avec la possibilité de les afficher ou non. La configuration du logiciel doit laisser a l'utilisateur la possibilité de spécifier la possibilité de réediter ou non le fichier source \textsc{Google Earth}. Ces zonnes seront représentée par une surface colorée en 2 dimention.
    \item[Les routes] Les routes devront être implémenté dans \textsc{Google Earth} avec la possibilité de les afficher ou non. La configuration du logiciel doit laisser a l'utilisateur la possibilité de spécifier la possibilité de réediter ou non le fichier source \textsc{Google Earth}. Cest route seront représentée par une ligne de couleur Jaune. Les points definissant cette route ne seront pas ilustré afin de ne pas faire de doublon avec les points caractéristique. Les coordonée des points de chaque route devront être defini a partir des points caractéristique en mémoire.
    \item[Les plan de vol] Les plans de vol devront être implémenté dans \textsc{Google Earth} avec la possibilité de les afficher ou non. La configuration du logiciel doit laisser a l'utilisateur la possibilité de spécifier la possibilité de réediter ou non le fichier source \textsc{Google Earth}. Les plan de vol doivent pouvoir être visualisé dans \textsc{Google Earth} en fonction du temps. Pour se faire une heure théorique de passage sera calculé par le programme pour chaque point definissant le plan de vol. Toutes les informations concernant chaque plan de vol tel que ca route, les points constituant sa route et sa situation dans le temps devront etre regroupé dans un dossier. Le message FPL de l'avion doit être visible dans la description de ce dossier. Les plans de vol seront visible durant toute la durée du vol et représenté par une ligne noire. La visualisation dans le temps sera représenté par un segment de couleur choisie aléatoirement pour chaque vol defini par les deux points les plus proche de l'heure en parametre dans le logiciel(un point avant et un point après). Ce segment et ses points ne serons visible qu'a partir de l'heure du premier points jusqu'a l'heure du deuxieme. 
    \item[l'itersection du plan de vol avec la zone \textsc{Aci}] L'intersetion, si elle a lieu, entre le plan de vol et la zone \textsc{Aci} doit être calculé, définie et representé dans \textsc{Google Earth} par un point rouge accompagné du nom de l'avion et de l'heure d'intersection affiché en rouge egalement. Ces points devront être contenu dans le dossier du concerné.
    \item[Les report \textsc{Ads}] Chaque report \textsc{Ads} serra composé de ces points de report ainsi que des points calculé par le système. Chaque report serra regroupé dans un dossier par vol et aura comme description le message recu. Chaque point calculé par le systeme sera attribué et regroupé avec le report précedent. Les vol sront representé par une ligne blanche retracant tout les report recu, ainsi que chaque point affiché dans le temps. L'interet etant de visualisé l'ecart entre le chemin parcouru par l'avion et le plan de vol déposé ainsi que la diferenec entre la trajectiore de l'avion et celle calculée par le système \textsc{Tiare}. 
\end{description}

















%Méthode employée à consommateur de personne à disposition, produit très riche si compétence,  adapté et performant, Evolution désordonnée si pas maitrisé (base de données en plus ), changement des spécifications  en cours de projet, difficulté de rédaction de spécification produit fini concentre sur le dev et moins sur la doc.
%Pas de rédaction de manuel d’utilisateur,

\section{La gestion de projet}
L'utilisation de l'extreme programming pour gérer le projet aura été réellement bénéfique. On notera tout de même que cette méthode requière des clients extrêmement compétents et réactifs. En effet sans compétence de la part du client le projet peu rapidement tourner en rond.

Le fait de renouveler sans cesse les besoins et spécifications permet de réaliser un produit riche, adapté et performant. L'évolution quant à elle demande une maîtrise bien plus stricte qu'avec une méthode de gestion de projet plus classique sous peine de devenir rapidement désordonnée.

Nous avons utilisé la méthode agile pour préciser les besoins et obtenir une spécification applicable à la concrétisation d'un vrai logiciel. Ainsi nous pouvons aussi dire que nous sommes restés au niveau 1 d'un cycle en V

\section{Le projet}
Nous avons atteint un grand nombre d'objectif avec ce projet, il nous est capable d'analyser des plans de vol et de les mettre en corrélation avec les reports reçu de l'avion par l'intermédiaire de liaison satellite.

Le projet n'est pas fini. Un grand nombre d'améliorations restent à implémenter tels qu'une interface graphique ou encore une base de donnée \nref{evolution}.

\section{Un stage formateur}
    \subsection{Un apport technique}
J'ai pu au cours de ce stage approfondir et mettre en pratique mes connaissances en programmation python. Mais j'ai surtout pu découvrir le monde de l'aéronautique. 

En effet ce stage ma permis de voir les technologies utilisées dans les zones de contrôles. J'ai donc pu prendre connaissance des technologies de détection des avions de dernière génération (des fois pas encore mis en place) tels que les radars secondaires Mode S. Ou encore les systèmes de détections de collisions. J'ai également pu m'instruire sur leurs solutions de télécommunications (Satellite, \textsc{Vhf}) qui ne sont pas enseignées sous cet angle (pratique et non théorique) dans le cadre de mon cursus.

J'ai aussi pu découvrir le fonctionnement d'un serveur \textsc{Ntp}, les principe de la para-virtualisation ou encore la mise en place de réseaux privé virtuel. Autant de domaines n'étant pas en relation direct avec le stage mais qui auront une grande utilité dans mon avenir professionnel. 

Un autre point découvert dans la pratique aura été des méthodes de gestion de projet. La première étant le cycle en V du fait que ce soit celle qui est appliqué au sein de l'entreprise. La deuxième étant l'extreme programing utilisée pour le projet du stage.

L'ouverture d'esprit du personnel faisant partie ou travaillant en sous-traitance pour la \textsc{Dgac} y a fortement contribué. 

    \subsection{Des rapport humains}
Ce stage m'a également permis, notamment lors des pauses cafés ou repas du midi, d'échanger avec un grand nombre d'ingénieurs travaillant pour la \textsc{Dgac} ou en sous-traitance. 

Ce sont grâce à ces échanges que j'ai pu acquérir une grande partie des connaissances en aéronautique et en informatique citées précédemment. En effet ces personnes n'ont pas héritées à prendre un peu de leur temps pour me faire des schémas sur le fonctionnement des différentes technologies de radars ou encore retrouver leur rapport de test (benchmark) réalisé sur différentes mise en place de virtualisations d'OS (Xen, Kvm, VMWere). 

\section{Conclusion}
Ce stage aura été une expérience professionnel très enrichissante. Il m'aura permis de découvrir le monde de l'aéronautique. Il m'aura aussi permis de découvrir plusieurs méthodes de gestion de projet ainsi que des connaissances techniques variées.

Au delà des aspects pédagogiques techniques, il m'aura aussi permis de me familiariser avec le monde de l'entreprise du point de vue d'un ingénieur. 

Le contexte m'aura aussi sensibilisé sur la qualité. En effet dans l'aéronautique la gestion de la qualité est des plus importante du fait qu'un système ne peut pas tomber en panne sous peine d'avoir de grave conséquences. 
 











\chapter{Evolution projet}
Ce projet est loin d'être arrivé a termes. Nous allons donc voir ici ce qui pourrait être fait afin de perfectionner ce logiciel. Les évolution seront axées sur trois points:
\begin{itemize}
    \item La mise en place d'une interface graphique.
    \item L'automatisation de l'acquisition.
    \item La pérennisation des données.
\end{itemize}

\section{La mise en place d'une interface graphique}
Comme il a été expliqué précédemment \nref{fonctionnement}, la configuration du logiciel est effectuer manuellement par l'intermédiaire de fichiers textes et son exécution est effectuée en ligne de commande. C'est pourquoi une interface graphique faciliterait grandement son utilisation.

Cette interface devrait pouvoir faciliter la configuration et l'exécution du programme, elle pourrait être basée sur des technologie web afin de la rendre portable tout en séparent le traitement des données de l'utilisation du fichier final dans \textsc{Google Earth}. En effet le programme pourrait être lancé a distance sur une machine, cela permettrait de sécuriser l'accès au données tout en libérant les ressources du poste de l'utilisateur.

Pour faciliter la configuration un histogramme avec tous les vol figurant entre deux date sélectionnée pourrait être réalisé, cela permettrait de mieux visualiser le trafic et de pouvoir cibler les vols a afficher.

Il pourrait aussi être intéressant d'inclure l'affichage final dans l'interface web, tout en laissant la possibilité a l'utilisateur de télécharger le fichier afin d'exploiter pleinement toutes les fonctionnalité du logiciel \textsc{Google Earth} tel que la mesure de distance entre deux points.

\section{L'automatisation de l'acquisition}
Actuellement chaque fichier à traiter est récupéré manuellement. On pourrait concevoir un système qui irait de lui même chercher les fichiers nécessaire dans le système \textsc{Tiare} et les mettre automatiquement à la disposition du programme.

\section{La pérennisation des données}
Dans une optique de pouvoir rejouer simplement des situations passées, on pourrait mettre en place un système de base de données légère tel que SQLite\footnote{SQLite est une bibliothèque écrite en C qui propose un moteur de base de données relationnelles accessible par le langage SQL.}. Contrairement aux serveurs de bases de données traditionnels, comme MySQL ou PostgreSQL, sa particularité est de ne pas reproduire le schéma habituel client-serveur mais d'être directement intégrée aux programmes. L'intégralité de la base de données (déclarations, tables, index et données) est stockée dans un fichier indépendant de la plate-forme.

Ce procédé couplé à un traitement automatique permettrait de mettre et garder en mémoire tout les vols disponibles sur le système \textsc{Tiare}. Il permettrais donc de pouvoir rejoué des situations qui n'on été enregistrées plusieurs mois avant. 





% Bibliographie
\bibliographystyle{plain}
\bibliography{tex/biblio}

% Début annexes
\appendix

\chapter{Annexes}
%%¯¯¯¯¯¯¯¯¯¯¯¯¯¯¯¯¯¯¯¯¯¯
\renewcommand{\baselinestretch}{1}\small \normalsize
%Prévoir une annexe avec la description de l’ADSC / CPDLC
\section{Codes sources du projet}
    \subsection{Manu\label{pyManu}}
Ce fichier sert à executer tout le programme:
\lstinputlisting{/home/manu/DTI/Manu.py}\newpage

    \subsection{Config\label{pyConfig}}
Nous avons ici le fichier de configuration. Celui ci sert notament à se passer temporairement d'une interface graphique.
\lstinputlisting{/home/manu/DTI/manu.cfg}\newpage

    \subsection{modules/Ads\label{pyAds}} 
%Met en mémoire les information sur les report \textsc{Ads}. \nref{pyAds}
Ce module lit le fichier de trace Ads du système tiraré et crée pour chaque aéronef un Objet Python ayant pour identifiant le nom de l'avion suivi de la date et l'heure. A cet objet est ensuite associé tout report lui concernant. Il recois donc les message recu par l'Ads-c et converti les point en coordonée. Mais il récupère aussi tout les points intermédiaire calculé par le système.
\lstinputlisting{/home/manu/DTI/modules/Ads.py}\newpage


    \subsection{modules/Aoi\label{pyAoi}} 
Ce module permet de définir tout les volumes utilisés pour concevoir les zone de contrôles. Il récuper dans le fichiers Asf chaque volume qu'il stocke dans un obejet comprenant chaque coordonée du volume aisin que ca tranche d'altitude.
\lstinputlisting{/home/manu/DTI/modules/Aoi.py}\newpage

    \subsection{modules/CharacteristicPoints\label{pyCP}}
Ce module met en mémoire tous les points remarquables disponible sur le système. Ces points seront ensuite utilisé pour concevoire les routes et les plans de vols.
\lstinputlisting{/home/manu/DTI/modules/CharacteristicPoints.py}\newpage

    \subsection{modules/Convertion\label{pyConvertion}} 
Regroupe plusieurs fonctions utiliser pour convertir des donnée.
\lstinputlisting{/home/manu/DTI/modules/Convertion.py}\newpage

    \subsection{modules/Fdp\label{pyFdp}}
Définit et met en mémoire toutes les zone de contrôles. Pour definir ces zones les volumes mis en mémoire a l'aide du module Aoi \nref{pyAoi}
\lstinputlisting{/home/manu/DTI/modules/Fdp.py}\newpage

    \subsection{modules/Fpl\label{pyFpl}}
définit et mets en mémoire les plans de vol. Il recupère tous les plan de vol dans les fichiers contenu dans le répertoire "source" et contenat "FPL" dans le nom.

Lorsque aucune date n'est renseigné dans le message Fpl une date est crée arbitrairement en fonction de la date et l'heure d'envoi du message et l'heure de décolage de l'avion.

les trames Fpl (expliqué a la ligne 60 de la source) sont sous la forme : \newline
(FPL-THT712-IX-A343/H-SXJIRYGWZ/SD-NTAA1630-N0479F400 DCT MOANA DCT TEANO DCT KARNO DCT 1755S14905W DCT PASTI DCT CORAL DCT OVINI DCT ONIDO DCT 18S149W/N0477F410 DCT DEBUT DCT FULL DCT FIN DCT BENKO/N0321F050 DCT TETIA DCT MANEV DCT BB/N0321F200 DCT IDUTA DCT-NTAA0345 NCRG-REG/FOSUN SEL/BMER DAT/SV NAV/RNP10 DLE/ONIDO 0061 DLE/BB 0027 RMK/CHARTER FLIGHT FOR ECLIPSE TRACK SOUTH OF TAHITI AND MEHETIA AT FL 410 EXPECT SIGHT SEEING REQUEST FOR DEPARTURE OVER MOOREA AND TAHITI AFTER THE ECLIPSE EXPECT SAME REQUEST UP TO TETIAROA AND BORA BORA BEFORE LANDING IN PPT EXPECT A 5000FT REQUEST OVER BORA BORA-E/0940 P/TBN R/VE S/M J/L D/8 440 C YELLOW A/BLUE/WHITE)

Toutes erreur est enregistrée dans un fichier de log.
\lstinputlisting{/home/manu/DTI/modules/Fpl.py}\newpage

    \subsection{modues/GetOfFiles\label{pyGOF}} 
Coordonne la récupération des donnée, c'est lui qui va chercher la configuration et lance les modules tel que Aoi, Fdp ou encore Fpl. 
\lstinputlisting{/home/manu/DTI/modules/GetOfFiles.py}\newpage

    \subsection{modules/\textsc{Kml}\label{pyKML}}
Afin de pouvoir réaliser les document KML, un module a été implémenté. Celui-ci a pour objectif de mettre en forme le document final. Il ne réalise aucun calcul. Lors de l'initiation une variable est instencié. Celle-ci accumulera toute la mise en forme du document jusqu'a l'appel de la foction de fin qui permettera de clore cette variable et de l'enregistrer dans un fichier texte.
\lstinputlisting{/home/manu/DTI/modules/KML.py}\newpage


    \subsection{modules/MakeKML\label{pyMakeKML}} 
C'est le module qui exploite toutes les données en mémoire et crée les fichiers \textsc{Kml}.
\lstinputlisting{/home/manu/DTI/modules/MakeKML.py}\newpage
 
    \subsection{modules/MakeKMZ\label{pyMakeKMZ}} 
Récupère les fichiers \textsc{Kml} pour les regrouper en un fichier \textsc{Kmz} plus maniable. \nref{pyMakeKMZ}
\lstinputlisting{/home/manu/DTI/modules/MakeKMZ.py}\newpage

    \subsection{modules/Routes\label{pyRoutes}} 
Définit et mets en mémoire les routes. Il utilise les points caractéristique précedement enregistrer pour associer les points de chaque route à des coordonées.
\lstinputlisting{/home/manu/DTI/modules/Routes.py}\newpage

    \subsection{modules/usualFonction\label{pyusualFonction}} 
Regroupe plusieurs fonction régulièrement utilisées. 
\lstinputlisting{/home/manu/DTI/modules/usualFonction.py}\newpage








\end{document}



